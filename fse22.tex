\PassOptionsToPackage{table,xcdraw}{xcolor}

\documentclass[sigconf,review,anonymous]{acmart}
\acmConference[ESEC/FSE 2022]{The 30th ACM Joint European Software Engineering Conference and Symposium on the Foundations of Software Engineering}{14 - 18 November, 2022}{Singapore}

%\documentclass[sigconf,review,anonymous]{acmart}
%\acmConference[ESEC/FSE 2021]{The 29th ACM Joint European Software Engineering Conference and Symposium on the Foundations of Software Engineering}{23 - 27 August, 2021}{Athens, Greece}

%\acmConference[ICSE 2022]{The 44th International Conference on Software Engineering}{May 21–29, 2022}{Pittsburgh, PA, USA}

%\documentclass[sigconf,review, anonymous]{acmart}
%\documentclass[sigconf]{acmart}

\usepackage{booktabs}   %% For formal tables:
                        %% http://ctan.org/pkg/booktabs
\usepackage{subcaption} %% For complex figures with subfigures/subcaptions
                        %% http://ctan.org/pkg/subcaption
\usepackage{array}
\usepackage{amsmath,amsfonts}
\usepackage{algorithm}
\usepackage[noend]{algpseudocode}
%\usepackage{algorithmic}
\usepackage{graphicx}
\usepackage{textcomp}
\usepackage{float}
\usepackage{listings}
\usepackage{xspace}
\usepackage{multirow}
\usepackage{amsthm}
\newtheorem{definition}{Definition}
\usepackage{balance}

\usepackage[skins]{tcolorbox}

\usepackage{xcolor,pifont}
\newcommand*\colourcheck[1]{%
	\expandafter\newcommand\csname #1check\endcsname{\textcolor{#1}{\ding{52}}}%
}
\colourcheck{blue}
\colourcheck{green}
\colourcheck{red}

\newtcolorbox{myframe}[2][]{%
  enhanced,colback=white,colframe=black,coltitle=black,
  sharp corners,
  toprule=1.0pt,
  rightrule=0.3pt,
  leftrule=0pt,
  bottomrule=0pt,
  fonttitle=\itshape\scshape\large,
  left=0pt,right=5pt,top=5pt,bottom=3pt,
  attach boxed title to top right={yshift=-0.3\baselineskip-0.4pt,xshift=-5mm},
  boxed title style={tile,size=minimal,left=0.2mm,right=0.5mm,
    colback=white,before upper=\strut},
  title=#2,#1
}

%\newcommand{\code}[1]{{\footnotesize\textsf{#1}}}

\newcommand{\tool}{\textsc{Tool}\xspace}

\newtheorem{Definition}{Definition}
\newtheorem{Claim}{Claim}
\newtheorem{Lemma}{Lemma}
\newtheorem{Theorem}{Theorem}

\newcolumntype{L}[1]{>{\raggedright\arraybackslash}p{#1}}
\newtheorem{observation}{Observation}
\newtheorem{property}{Property}
\newcommand{\code}[1]{{\footnotesize\texttt{#1}}}
\usepackage{amsthm}
 \definecolor{dkgreen}{rgb}{0,0.6,0}
\definecolor{gray}{rgb}{0.5,0.5,0.5}
\definecolor{mauve}{rgb}{0.58,0,0.82}
\lstset{frame=tb,
  language=Java,
  aboveskip=3mm,
  belowskip=3mm,
  showstringspaces=false,
  columns=flexible,
  basicstyle={\small\ttfamily},
  numbers=left,
  numberstyle=\tiny\color{gray},
  keywordstyle=\color{blue},
  commentstyle=\color{dkgreen},
  stringstyle=\color{mauve},
  breaklines=true,
  breakatwhitespace=true,
  tabsize=4
}



\begin{document}

%\title[{\tool}: Deep Fault Localization with Code Coverage Representation Learning]{{\tool}: Deep Fault Localization with Code Coverage Representation Learning}

\title[Title Goes Here]{Title Goes Here}

%Context-aware
%ML
%Clone-aware
%Graph-based


%%%---- AUTHORS BLOCK ------

\setcopyright{none}

\settopmatter{printacmref=false, printfolios=false}

\renewcommand\footnotetextcopyrightpermission[1]{} % removes footnote with conference information in first column


%(1) present information sorted in a way that a CNN can "see" patterns
%discriminating between faulty and non faulty statements more easily;

%(2) identify the actual crash statement to the network;

%(3) present more information to the deep neural network in the form of
%a summary of data dependences for each statement as well as source
%embedding; and

%(4) the suspiciousness of a statement is seen taking into account
%relationships to other statement, as opposed to a statement by itself”



%\input{sections/abstract}
\begin{abstract}
Abstract goes here ...
\end{abstract}


%\settopmatter{printacmref=true, printccs=true, printfolios=false}

%\begin{CCSXML}
%<ccs2012>
%<concept>
%<concept_id>10011007.10011006.10011073</concept_id>
%<concept_desc>Software and its engineering~Software maintenance tools</concept_desc>
%<concept_significance>500</concept_significance>
%</concept>
%</ccs2012>
%\end{CCSXML}

%\ccsdesc[500]{Software and its engineering~Software maintenance tools}

%\keywords{Deep Learning; Automated Program Repair; Context-based Code Transformation Learning}


\maketitle

\section{Introduction}
\label{intro:sec}

During software evolution, developers make several changes over time
and commit them into a source code repository. The changes to the
source files that are committed to the repository at the same
transaction are often referred to as a {\em change set} or a {\em
  commit}. In an ideal world, each commit should be about one purpose
or concern regarding the programming task at hand.  Unfortunately, Tao
{\em et al.}~\cite{tao-fse12}, Kim {\em et
  al.}~\cite{kim-emse16,kim-msr13}, and Hill {\em et
  al.}~\cite{hill-tse12} have reported that many change sets or
commits tangle different concerns including the changes for
bug-fixing, refactoring, enhancements/improvements, or
documentation. Such change sets are called {\em tangled code changes}
or {\em tangled commits}~\cite{kim-emse16,kim-msr13}. The prior work
reported two reasons for tangled commits: time pressure in committing
the changes, and unclear boundaries between the concerns for code
changes~\cite{flexeme-fse20}.

Tangled commits pose several issues in software development. First,
they affect software quality in both hampering program
comprehension~\cite{tao-fse12} and reducing separation of concerns in
code changes~\cite{flexeme-fse20}. Second, the tangled commits
might contain the bug-fixing changes for one bug that are mixed with
the fixes for other bugs as well as different types of changes for
refactoring, enhancements, or
documentation~\cite{kim-emse16,kim-msr13,nguyen-issre13}. Those
tangled commits have negative impacts on the accuracy of bug
prediction or bug localization models that rely on the data mined from
the version histories~\cite{kim-emse16,kim-msr13}. Those models 
consider an entire commit as for fixing or non-fixing, thus,
are significantly affected by the tangled commits.

Recognizing the need of the tools that untangle, i.e., decompose a
commit into untangle changes, several researchers have proposed
different approaches that can be broadly classified into two
categories: {\em mining software repositories}, and {\em program
  analysis}.

First, earlier approaches leverage the {\em mining software
  repositories (MSR)} techniques. Herzig {\em et
  al.}~\cite{kim-msr13,kim-emse16} utilize a confidence voter
technique together with agglomerative clustering on the change
operations to untangle the commits.
%Each confidence voter is responsible for an important aspect
%including call-graphs, change couplings, data dependencies, and
%distance measures.
However, the voters are independent, thus, do not reflect well the
interdependency nature of program elements under change. In contrast,
Kirinuki {\em et al.}~\cite{higo-apsec16, higo-icpc14} rely on the
histories of the co-changes to split the tangled code changes before
they are committed. However, they do not consider the dependencies
among the changes such as data or control dependencies. Dias {\em et
  al.}~\cite{dias-saner15} also use confidence voters, but on the
fine-grained change events in an IDE. The scores are converted into
the similarity ones via a Random Forest Regressor, which are used in
the agglomerative clustering to partition the tangled changes.  The
second category of untangling approaches leverage the {\em static
  analysis} techniques. Roover {\em et al.}~\cite{roover-scam18} use
program slicing to segment a commit across a Program Dependency Graph
(PDG).  However, they are limited handling interprocedural and
cross-file dependencies. Barnett {\em et al.}~\cite{barnett-icse15}
utilize def-use chains, and cluster them. If the def-use chain all
fall into a method, it is considered as trivial, otherwise,
non-trivial. Because igoring the trivial clusters, it can miss tangled
concerns. To improve over that, Flexeme~\cite{flexeme-fse20} uses
multi-version PDG augmented with name/lexeme flows in the edges, and
applies Agglomerative Clustering using Graph Similarity on the graph
to untangle its commits.


\newpage

\balance

%\bibliographystyle{plain}
%\bibliographystyle{ACM-Reference-Format}
\bibliographystyle{ACM-Reference-Format}

\bibliography{References}

\end{document}
