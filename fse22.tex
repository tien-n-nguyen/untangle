\PassOptionsToPackage{table,xcdraw}{xcolor}

\documentclass[sigconf,review,anonymous]{acmart}
\acmConference[ESEC/FSE 2022]{The 30th ACM Joint European Software Engineering Conference and Symposium on the Foundations of Software Engineering}{14 - 18 November, 2022}{Singapore}

%\documentclass[sigconf,review,anonymous]{acmart}
%\acmConference[ESEC/FSE 2021]{The 29th ACM Joint European Software Engineering Conference and Symposium on the Foundations of Software Engineering}{23 - 27 August, 2021}{Athens, Greece}

%\acmConference[ICSE 2022]{The 44th International Conference on Software Engineering}{May 21–29, 2022}{Pittsburgh, PA, USA}

%\documentclass[sigconf,review, anonymous]{acmart}
%\documentclass[sigconf]{acmart}

\usepackage{booktabs}   %% For formal tables:
                        %% http://ctan.org/pkg/booktabs
\usepackage{subcaption} %% For complex figures with subfigures/subcaptions
                        %% http://ctan.org/pkg/subcaption
\usepackage{array}
\usepackage{amsmath,amsfonts}
\usepackage{algorithm}
\usepackage[noend]{algpseudocode}
%\usepackage{algorithmic}
\usepackage{graphicx}
\usepackage{textcomp}
\usepackage{float}
\usepackage{listings}
\usepackage{xspace}
\usepackage{multirow}
\usepackage{amsthm}
\newtheorem{definition}{Definition}
\usepackage{balance}

\usepackage[skins]{tcolorbox}

\usepackage{xcolor,pifont}
\newcommand*\colourcheck[1]{%
	\expandafter\newcommand\csname #1check\endcsname{\textcolor{#1}{\ding{52}}}%
}
\colourcheck{blue}
\colourcheck{green}
\colourcheck{red}

\newtcolorbox{myframe}[2][]{%
  enhanced,colback=white,colframe=black,coltitle=black,
  sharp corners,
  toprule=1.0pt,
  rightrule=0.3pt,
  leftrule=0pt,
  bottomrule=0pt,
  fonttitle=\itshape\scshape\large,
  left=0pt,right=5pt,top=5pt,bottom=3pt,
  attach boxed title to top right={yshift=-0.3\baselineskip-0.4pt,xshift=-5mm},
  boxed title style={tile,size=minimal,left=0.2mm,right=0.5mm,
    colback=white,before upper=\strut},
  title=#2,#1
}

%\newcommand{\code}[1]{{\footnotesize\textsf{#1}}}

\newcommand{\tool}{\textsc{UTango}\xspace}

\newcommand{\mvpdg}{$\delta$-PDG$^{i,j}$}

\newtheorem{Definition}{Definition}
\newtheorem{Claim}{Claim}
\newtheorem{Lemma}{Lemma}
\newtheorem{Theorem}{Theorem}

\newcolumntype{L}[1]{>{\raggedright\arraybackslash}p{#1}}
\newtheorem{observation}{Observation}
\newtheorem{property}{Property}
\newcommand{\code}[1]{{\footnotesize\texttt{#1}}}
\usepackage{amsthm}
 \definecolor{dkgreen}{rgb}{0,0.6,0}
\definecolor{gray}{rgb}{0.5,0.5,0.5}
\definecolor{mauve}{rgb}{0.58,0,0.82}
\lstset{frame=tb,
  language=Java,
  aboveskip=3mm,
  belowskip=3mm,
  showstringspaces=false,
  columns=flexible,
  basicstyle={\small\ttfamily},
  numbers=left,
  numberstyle=\tiny\color{gray},
  keywordstyle=\color{blue},
  commentstyle=\color{dkgreen},
  stringstyle=\color{mauve},
  breaklines=true,
  breakatwhitespace=true,
  tabsize=4
}



\begin{document}

%\title[{\tool}: Deep Fault Localization with Code Coverage Representation Learning]{{\tool}: Deep Fault Localization with Code Coverage Representation Learning}

\title[Untangling Commits with Context-aware, Graph-based, Code Change Clustering Learning Model]{{\tool}: Untangling Commits with Context-aware,\\ Graph-based, Code Change Clustering Learning Model}

%Context-aware
%ML
%Clone-aware
%Graph-based


%%%---- AUTHORS BLOCK ------

\setcopyright{none}

\settopmatter{printacmref=false, printfolios=false}

\renewcommand\footnotetextcopyrightpermission[1]{} % removes footnote with conference information in first column


%(1) present information sorted in a way that a CNN can "see" patterns
%discriminating between faulty and non faulty statements more easily;

%(2) identify the actual crash statement to the network;

%(3) present more information to the deep neural network in the form of
%a summary of data dependence for each statement as well as source
%embedding; and

%(4) the suspiciousness of a statement is seen taking into account
%relationships to other statement, as opposed to a statement by itself”



%\input{sections/abstract}
\begin{abstract}
During software evolution, developers make several changes and commit
them into the repositories. Unfortunately, many of them tangle
different purposes, both hampering program comprehension and reducing
separation of concerns. Automated approaches with deterministic
solutions have been proposed to untangle commits.
%Unlike the state-of-the-art, deterministic untangling approaches,

In this work, we present {\bf \tool}, a machine learning (ML)-based
approach that learns to untangle the changes in a commit.
%
We develop a {\em novel code change clustering learning model} that
learns to cluster the code changes, represented by the embeddings,
into different groups with different concerns.
%
We adapt the agglomerative clustering algorithm into a
supervised-learning clustering model operating on the learned code
change embeddings via trainable parameters and a loss function in
comparing the predicted clusters and the correct ones during training.
%
%Another key idea of {\tool} is
To facilitate our clustering learning model, we develop a {\em
context-aware, graph-based, code change representation learning
model}, leveraging Label, Graph-based Convolution Network to produce
the {\em contextualized embeddings for code changes}, that integrates
program dependencies and the surrounding contexts of the changes. The
contexts and cloned code are also explicitly represented, helping
{\tool} distinguish their concerns.
%
Our empirical evaluation on a public C\# dataset
with 1,612 tangled commits shows that it achieves the
accuracy of 28.6\%--462.5\%, relatively higher than the
state-of-the-art approaches in clustering the changed code. We also
evaluated {\tool} in a Java dataset with 14k+ tangled commits. The result shows
that it achieves 13.3\%-100.0\% relatively higher accuracy than the
state-of-the-art approaches.

%Our sensitivity also shows that all designed components in {\tool}
%contributes positively to its high accuracy.}
\end{abstract}

%\textcolor{red}{Our empirical evaluation on a real-world C\# dataset
%with 21k commits and 1,612 concerns shows that it achieves the
%accuracy of XX.X\%--XX.X\% and YY.Y\%--YY.Y\% relatively higher than
%the state-of-the-art approaches in cross-project and within-project
%settings. We also evaluated {\tool} in a Java dataset with XX
%commits. The results show that it achieves accuracy XX.X\% relatively
%higher than the state-of-the-art approach.  Our sensitivity also shows
%that all designed components in {\tool} contributes positively to its
%high accuracy.}

%We develop a novel code change representation learning model
%leveraging Graph-based Convolution Network to build the vector
%representations for the changes, which integrate program dependencies,
%the surrounding contexts of the changes, as well as the code clone
%relationships.

%\settopmatter{printacmref=true, printccs=true, printfolios=false}

%\begin{CCSXML}
%<ccs2012>
%<concept>
%<concept_id>10011007.10011006.10011073</concept_id>
%<concept_desc>Software and its engineering~Software maintenance tools</concept_desc>
%<concept_significance>500</concept_significance>
%</concept>
%</ccs2012>
%\end{CCSXML}

%\ccsdesc[500]{Software and its engineering~Software maintenance tools}

%\keywords{Deep Learning; Automated Program Repair; Context-based Code Transformation Learning}


\maketitle

\section{Introduction}
\label{intro:sec}

During software evolution, developers make several changes over time
%and commit them into a source code repository.
to perform software maintenance tasks. The changes to the
source files that are committed to the repository in the same
transaction are referred to as a {\em change set} or a {\em
  commit}. For separation of concerns, each commit should be about one
purpose~or~concern regarding the programming task at hand.
Unfortunately, it has been reported that many commits tangle different
concerns including the changes for bug-fixing, refactoring,
enhancements, improvements, or
documentation~\cite{tao-fse12,kim-emse16,kim-msr13,hill-tse12,nguyen-issre13}.
%Tao {\em et al.}~\cite{tao-fse12}, Kim {\em et
%  al.}~\cite{kim-emse16,kim-msr13}, and Hill {\em et
%  al.}~\cite{hill-tse12} have reported that many change sets tangle
%different concerns including the changes for bug-fixing, refactoring,
%enhancements/improvements, or documentation.
Such change sets are called {\em tangled code changes} or {\em tangled
  commits}~\cite{kim-emse16,kim-msr13}. The prior work reported two
reasons for tangled commits from developers' perspective: time
pressure in committing the changes, and unclear relations between the
concerns for code changes~\cite{flexeme-fse20}.

Tangled commits pose several issues in software development. First,
they affect software quality as they both hamper program
comprehension~\cite{tao-fse12} and reduce the separation of concerns
in code changes~\cite{flexeme-fse20}. Second, the tangled commits
might contain the bug-fixing changes for one bug that are mixed with
the fixes for other bugs as well as different types of changes for
refactoring, enhancements, or
documentation~\cite{kim-emse16,kim-msr13,nguyen-issre13}. Those
tangled commits have negative impacts on the accuracy of bug
prediction or bug localization models that rely on the changes mined from
the repository~\cite{kim-emse16,kim-msr13}. Those models
consider an entire commit as for fixing or non-fixing, thus, are
significantly affected by the tangled commits.

Recognizing the need of the tools that untangle, i.e., decompose a
commit into untangle changes, several researchers have proposed
different approaches that can be broadly classified into two
categories: {\em mining software repositories}, and {\em program
  analysis}.

First, earlier approaches leverage {\em mining software repositories
  (MSR)} techniques to untangle commits. Herzig {\em et
  al.}~\cite{kim-msr13,kim-emse16} utilize a confidence voter
technique together with agglomerative clustering on the change
operations for untangling.
%the commits.
%Each confidence voter is responsible for an important aspect
%including call-graphs, change couplings, data dependencies, and
%distance measures.
However, the voters are independent, thus, do not reflect well the
interdependency nature of program elements under change.
%
Kirinuki {\em et al.}~\cite{higo-apsec16, higo-icpc14} consider a
commit as tangled if it includes another commit in the past. However,
there are other tangled commits whose part of them have~not occurred
in the past.
%rely on the histories of the co-changes to split the tangled code
%changes before they are committed. However, they do not consider the
%relations among the changes such as data or control dependencies.
%
Dias {\em et al.}~\cite{dias-saner15} use confidence voters
on the fine-grained change events in an editor. The scores are con\-verted
into the similarity ones via a Random Forest Regressor, which are used
in agglomerative clustering to partition tangled changes.

The second category of untangling approaches leverage the {\em static
  analysis} techniques. Roover {\em et al.}~\cite{roover-scam18} use
program slicing to segment a commit across a Program Dependency Graph
(PDG).  However, it is limited in handling interprocedural and
cross-file dependencies. Barnett {\em et al.}~\cite{barnett-icse15}
use def-use chains, and cluster them. If the def-use chain all falls
into a method, it is considered as trivial, otherwise,
non-trivial. Because igoring the trivial clusters, it can miss tangled
concerns. Flexeme~\cite{flexeme-fse20} uses multi-version PDG
augmented with name flows in the edges, and applies agglomerative
clustering using graph similarity on that graph to untangle the
commits. SmartCommit~\cite{smartcommit-fse21} uses a
graph-partitioning algorithm on a graph representation to capture the
relations among code changes (hard and soft links, refactoring links,
cosmetic links, etc.).

%the targets of their clustering algorithms are either changes, change
%operations, change events, or slices, PDGs, which do not sufficiently
%contain the information for untangling commits. The reason is that
%the boundaries across the concerns in a commit do not neccessarily
%and natually map to a specific clustering criteria on those targets
%such as changes or PDGs. First, the boundaries across the concerns
%in a commit do not neccessarily and natually map to a clustering
%criteria on the PDG (with/without name flows).

%The concerns might be linked with multiple edges, and a statement
%might belong to multiple concerns.

%These points make the clustering algorithms difficult to specify the
%clustering criteria to achieve the best partitioning.

Despite their successes, the state-of-the-art untangling commit
techniques still have limitations. First, the boundaries across the
concerns in a commit do not neccessarily and natually map to
clustering criteria of a clustering algorithm running on the PDG
(with/without name flows), program slices, change operations, or the
changes themselves. The concerns might be linked via multiple edges,
and a statement might belong to multiple concerns. Applying a
clustering algorithm on the PDG, slices, or change graphs makes it
difficult to specify clustering criteria to achieve the partitions
matching with the concerns.
%
Second, the goal is to decompose the changes in a commit. However, the
existing approaches {\em do~not~consider a change w.r.t. the context of
surrounding code~with a clear distinction of the changed elements
and the un-changed, contextual ones}. Such context could help
distinguish the concerns for the changes. Finally, not all the changes
in the same concern need to have program dependencies among one
another. The logic connection among the co-changed code in the same
commit could be due to the reasons different than program
dependencies. For example, two pieces of cloned code realizing the
same bubble sorting algorithm have the same bug, e.g., at the
comparison operator. They might be changed in the same commit to fix
that same logic bug.

%We develop {\tool}, a {\em novel code change clustering learning
%model} that learns to cluster the code changes, represented by the
%embeddings, into different groups with different concerns. The core of
%{\tool} is {\em context-aware, graph-based, code change representation
%learning (RL) model} leveraging Graph-based Convolution Network to
%produce the {\em contextualized embeddings (vectors)} for the code
%changes, that integrates program dependencies, the surrounding
%contexts of the changes, as well as the code clone relations. The
%contexts of the changes and cloned code are explicitly represented,
%helping the model distinguish their concerns/purposes.

%a machine learning-based approach that learns to untangle the changes
%in a commit into different clusters for different concerns.

%{\tool} is designed with the key ideas to address the above
%challenges.

%{\tool} represents the changes and the surrounding code via the
%multi-version program dependency graph, $\delta$-PDG (adapted from
%Flexeme~\cite{flexeme-fse20}) and uses the Graph Convolutional Network
%(GCN) to model the statements and their interdependencies in
%$\delta$-PDG.

%Finally, we design a novel \underline{code change representation learning}
%that integrates the graph structures in GCN for program dependencies,
%the representation of a change and its context, and the cloned code
%into the vector representations (i.e., embeddings) for the code
%changes in a commit.

%Tien
To address those challenges, we propose {\tool}, a {\bf novel code
  change clustering learning model} that learns to untangle a commit
by clustering the code changes (represented by the embeddings) into
different groups for different concerns.

While deterministic clustering criteria on the PDG, slices, or change
operations do not always produce the clusters that naturally map to
the boundaries between the concerns, a machine learning (ML) model is
expected to learn to cluster the changes, thus, untangling a
commit. The model can learn from the changes belonging to the same
concerns in the version history. To facilitate the code change
clustering learning, we develop a {\bf context-aware, graph-based,
  code change representation learning (RL) model}, leveraging
Graph-based Convolution Network (GCN) to produce the {\em contextualized
  embeddings (vectors) for the code changes}.

Our clustering learning model and context-aware, graph-based RL model
for code changes have the following unique characteristics that
facilitates the untangling of code changes. First, we use GCN to model
the changes and surrounding code by integrating both the versions
before and after the changes in a multi-version program dependence
graph, $\delta$-PDG~\cite{flexeme-fse20} that encodes the {\em
  \underline{program dependencies} between the changed and unchanged
  statements}. Second, to decompose the changes, we {\em explicitly
  represent the surrounding \underline{code context} of each
  change}. The explicit representation of the context could help
{\tool} learn the important features of a change (e.g., code
structures, data/control dependencies) to distiguish its concern among
others. Third, {\tool} also considers an important feature for each
change, that is, the cloned code that is similar to the code under
consideration. The idea is that the {\em two \underline{cloned code}
  with similar logic might be changed in the same manner in the same
  concern in a commit}. Fourth, to untangle a commit, in our code
change clustering learning model, agglomerative clustering is applied
on the {\em \underline{contextualized embeddings for the code
    changes}} that integrates richer, encoded information than the PDG
or program slices, thus, helping better distinguish the concerns of
the changes. Finally, we {\em \underline{adapt the agglomerative
    clustering algorithm}} into a {\em
  \underline{super}-\underline{vised-learning clustering model}} with
trainable parameters and a loss function to adjust the parameters by
comparing the predicted clusters and the correct ones during training.

%We then apply agglomerative clustering on the vectors to produce the
%final clusters to untangle the commit. We expect that agglomerative
%clustering will be more effective as running on the embeddings with
%richer, encoded information than on the PDG or slices in
%distinguishing the concerns of the changes.

  %\textcolor{red}{We conduct several experiments to evaluate {\tool}...}

  \textcolor{red}{We have conducted several experiments to evaluate
    {\tool}. Our experimental results on a real-world C\# dataset with
    21k commits and 1,612 concerns show that {\tool} achieves the
    accuracy of XX.X\%, XX.X\%, and XX.X\% relatively higher than the
    baseline approaches Flexeme~\cite{flexeme-fse20}, Barnett
    {\em et al.}~\cite{barnett-icse15}, and Herzig {\em et
      al.}~\cite{kim-emse16}, respectively. {\tool} can correctly
    cluster 39\% the changed statements into the correct
    concerns.
%
  We also evaluated {\tool} in a Java dataset with XX commits and
  X,XXX concerns. The results show that {\tool} achieves XX.X\%
  accuracy relatively higher than the state-of-the-art approach
  SmartCommit~\cite{smartcommit-fse21}. Our sensitivity analysis shows
  that all designed components in {\tool} contributes positively to
  its high accuracy. We also show that the changed statements in the
  same concerns are projected nearer to each other~than the ones in
  different concerns. The same changed statements in
  different concerns are projected farther away in the vector space.}

The key contributions of this work include:

{\bf 1. {\tool}: an ML-based commit-untangling approach with a novel
  code change clustering learning model.} It is the first ML model
that learns to untangle the commit by learning to cluster the code
changes.  {\tool} learns from the changes belonging to the same
concern in the version history. We adapt agglomerative clustering
into a supervised-learning clustering model.

{\bf 2. A Novel context-aware, graph-based representation learning for
  code changes.} We design GCN-based model to produce the {\em
  contextualized embeddings for the code changes}, that
integrates the program dependencies, the representations of changes
and contexts. Cloned code is also considered in untangling commits.

{\bf 3. Extensive empirical evaluation.} We evaluated {\tool} against
the recent approaches for untangling commits to show its
performance. Our tool and data are available at~\cite{utango-website}.

%\textcolor{red}{We have conducted several experiments to evaluate
%    {\tool}. Our experimental results on a real-world C\# dataset with
%    21k commits and 1,612 concerns show that {\tool} achieves the
%    accuracy of XX.X\%--XX\% and YY.Y\%--YY.Y\% relatively higher than
%    the state-of-the-art approaches Flexeme~\cite{flexeme-fse20},
%    Barnett {\em et al.}~\cite{barnett-icse15}, and Herzig {\em et
%      al.}~\cite{kim-emse16}, in cross-project and within-project
%    settings, respectively. {\tool} can correctly untangle the commits
%    by correctly clustering 39\% the changed statements into the
%    correct clusters/concerns.}


\section{Motivation}
\label{motiv:sec}

\subsection{Motivating Examples}
\label{exe:sec}



\begin{figure}[t]
	\centering
	\lstset{
		numbers=left,
		numberstyle= \tiny,
		keywordstyle= \color{blue!70},
		commentstyle= \color{red!50!green!50!blue!50},
		frame=shadowbox,
		rulesepcolor= \color{red!20!green!20!blue!20} ,
		xleftmargin=1.5em,xrightmargin=0em, aboveskip=1em,
		framexleftmargin=1.5em,
                numbersep= 5pt,
		language=Java,
    basicstyle=\scriptsize\ttfamily,
    numberstyle=\scriptsize\ttfamily,
    emphstyle=\bfseries,
                moredelim=**[is][\color{red}]{@}{@},
		escapeinside= {(*@}{@*)}
	}
	\begin{lstlisting}[]
private static IEnumerable<Tuple<SpecificationProperty, Maybe<Error>>> MapValuesImpl(...) {
  ...
  var pt = specProps.First();
(*@{\color{red}{   - var taken = values.Take(pt.Specification.CountOfMaxNumberOfValues().}@*) (*@{\color{red}{MapValueOrDefault(n => n, values.Count()));}@*)
(*@{\color{cyan}{   + var taken = values.Take(pt.Specification.CountOfMaxNumberOfValues().}@*) (*@{\color{cyan}{GetValueOrDefault(values.Count()));}@*)
   if (taken.Empty()) {...} ...
}
//--------------------------------------------------------------------------
public static TypeDescriptor WithNextValue(this TypeDescriptor descriptor, Maybe<TypeDescriptor> nextValue) {
(*@{\color{red}{  - return TypeDescriptor.Create(descriptor.TargetType,descriptor.MaxItems,}@*) (*@{\color{red}{nextValue.MapValueOrDefault(n => n, default(TypeDescriptor)));}@*)
(*@{\color{cyan}{  + return TypeDescriptor.Create(descriptor.TargetType,descriptor.MaxItems,}@*) (*@{\color{cyan}{nextValue.GetValueOrDefault(default(TypeDescriptor)));}@*)
}
//--------------------------------------------------------------------------
public static ParserResult<T> Build<T>(...) {
  ...
  select
(*@{\color{red}{- sp.Value.MapValueOrDefault(v=>v,}@*) (*@{\color{red}{sp.Specification.DefaultValue.MapValueOrDefault(d => d,}@*)
(*@{\color{cyan}{  + sp.Value.GetValueOrDefault(}@*)
(*@{\color{cyan}{  + \quad sp.Specification.DefaultValue.GetValueOrDefault(}@*)
(*@{\color{cyan}{   \quad  sp.Specification.ConversionType.CreateDefaultForImmutable()))).ToArray();}@*)
  var immutable = (T)ctor.Invoke(values);
  return immutable;
}
	\end{lstlisting}
        \vspace{-15pt}
        \caption{Co-Changed Statements for the Same Concern}
        \vspace{-6pt}
        \label{fig:motiv-cc}
\end{figure}

%public static ParserResult<T> Build<T>(...) {
%  ...
%  select
%(*@{\color{red}{  - sp.Value.MapValueOrDefault(}@*)
%(*@{\color{red}{  - \quad   v => v,}@*)
%(*@{\color{red}{  - \quad  sp.Specification.DefaultValue.MapValueOrDefault(}@*)
%(*@{\color{red}{  - \quad \quad   d => d,}@*)
%(*@{\color{cyan}{  + sp.Value.GetValueOrDefault(}@*)
%(*@{\color{cyan}{  + \quad sp.Specification.DefaultValue.GetValueOrDefault(}@*)
%(*@{\color{cyan}{   \quad  sp.Specification.ConversionType.CreateDefaultForImmutable()))).ToArray();}@*)
%  var immutable = (T)ctor.Invoke(values);
%  return immutable;
%}

Let us present the real-world examples and our observations for
motivation. Figure~\ref{fig:motiv-cc} shows an example from the
experimental dataset in a prior work~\cite{flexeme-fse20}. This commit
has three changed statements at lines 4, 10, and 17 of the methods
\code{MapValuesImpl}, \code{WithNextValue}, and \code{Build}. The
method \code{MapValueOrDefault} was renamed to
\code{GetValueOr\-Default}, causing the changes to the call sites.
These changes are deemed by the developers as serving to the same
concern/purpose of enhancing type resolution in the parser. These
statements have been changed for the same concern in a past
commit. This motivates us to build a machine learning (ML) model to
learn from the co-changes for the same concern in the version history
to untangle the current commit. For this example, the approaches
relying on the PDGs or program slices within individual methods
(e.g.,~\cite{flexeme-fse20},~\cite{roover-scam18}) cannot cluster the
changed statements into a group.

\noindent {\bf Observation 1.} {\em History of the co-changed statements
for the same concern could be a good source for an ML model to learn
untangle the current commits}.

\begin{figure}[t]
	\centering
	\lstset{
		numbers=left,
		numberstyle= \tiny,
		keywordstyle= \color{blue!70},
		commentstyle= \color{red!50!green!50!blue!50},
		frame=shadowbox,
		rulesepcolor= \color{red!20!green!20!blue!20} ,
		xleftmargin=1.5em,xrightmargin=0em, aboveskip=1em,
		framexleftmargin=1.5em,
                numbersep= 5pt,
		language=Java,
    basicstyle=\scriptsize\ttfamily,
    numberstyle=\scriptsize\ttfamily,
    emphstyle=\bfseries,
                moredelim=**[is][\color{red}]{@}{@},
		escapeinside= {(*@}{@*)}
	}
	\begin{lstlisting}[]
public override void Initialize() {
  ...
  // set algorithm framework models
(*@{\color{red}{- PortfolioSelection=new}@*) (*@{\color{red}{ManualPortfolioSelectionModel(QuantConnect.Symbol.Create("BTCUSD",}@*) (*@{\color{red}{SecurityType.Crypto, Market.GDAX));}@*)   
(*@{\color{cyan}{+ UniverseSelection=new}@*) (*@{\color{cyan}{ManualUniverseSelectionModel(QuantConnect.Symbol.Create("BTCUSD",}@*) (*@{\color{cyan}{SecurityType.Crypto, Market.GDAX));}@*)
  Alpha = new ConstantAlphaModel(InsightType.Price, InsightDirection.Up,...);
  PortfolioConstruction = new EqualWeightingPortfolioConstructionModel();
}
//--------------------------------------------------------------------------
public override void Initialize() {
  ...
  // set algorithm framework models
(*@{\color{red}{- PortfolioSelection=new}@*) (*@{\color{red}{ManualPortfolioSelectionModel(QuantConnect.Symbol.Create("SPY",}@*) (*@{\color{red}{SecurityType.Equity, Market.USA));}@*)   
(*@{\color{cyan}{+ UniverseSelection=new}@*) (*@{\color{cyan}{ManualUniverseSelectionModel(QuantConnect.Symbol.Create("SPY",}@*) (*@{\color{cyan}{SecurityType.Equity, Market.USA);}@*)
  Alpha = new ConstantAlphaModel(InsightType.Price, InsightDirection.Up,...);
  PortfolioConstruction = new EqualWeightingPortfolioConstructionModel();
//--------------------------------------------------------------------------
public override void Initialize() {
  ...
(*@{\color{red}{- PortfolioSelection = new CustomFundamentalPortfolioSelectionModel();}@*)
(*@{\color{cyan}{+ UniverseSelection = new CustomFundamentalUniverseSelectionModel();}@*)
  Alpha = new MacdAlphaModel(TimeSpan.FromMinutes(10),...);
  PortfolioConstruction = new EqualWeightingPortfolioConstructionModel();
}
	\end{lstlisting}
        \vspace{-15pt}
        \caption{Cloned Code in the Same Concern}
        \vspace{-6pt}
        \label{fig:motiv-clone}
\end{figure}

Figure~\ref{fiv:motiv-clone} shows an example of the cloned codes
that were modified in the same manner to ...

\begin{figure}[t]
	\centering
	\lstset{
		numbers=left,
		numberstyle= \tiny,
		keywordstyle= \color{blue!70},
		commentstyle= \color{red!50!green!50!blue!50},
		frame=shadowbox,
		rulesepcolor= \color{red!20!green!20!blue!20} ,
		xleftmargin=1.5em,xrightmargin=0em, aboveskip=1em,
		framexleftmargin=1.5em,
                numbersep= 5pt,
		language=Java,
    basicstyle=\scriptsize\ttfamily,
    numberstyle=\scriptsize\ttfamily,
    emphstyle=\bfseries,
                moredelim=**[is][\color{red}]{@}{@},
		escapeinside= {(*@}{@*)}
	}
	\begin{lstlisting}[]
public void mousePressed(MouseEvent evt) { 
  ...//SelectionTool.java
(*@{\color{red}{- if (figure != null) {}@*)
(*@{\color{cyan}{+ if (figure != null}@*) && (*@{\color{cyan}{figure.isSelectable()) {}@*)
       newTracker = createDragTracker(figure);
    } else {
       if (! evt.isShiftDown()) {...}
}
//--------------------------------------------------------------------------
protected void updateHoverHandles(DrawingView view, Figure f) {
  ...// SelectAreaTracker.java
  figure = f;
(*@{\color{red}{- if (figure != null) {}@*)
(*@{\color{cyan}{+ if (figure != null}@*) && (*@{\color{cyan}{figure.isSelectable()) {}@*)
      hoverHandles.addAll(figure.createHandles(-1));
    ...
}
	\end{lstlisting}
        \vspace{-15pt}
        \caption{Context}
        \vspace{-6pt}
        \label{fig:motiv-context}
\end{figure}





\section{Approach Overview}
\label{overview:sec}

\subsection{Important Concepts}
\label{concepts:sec}

\begin{Definition}[Program Dependence Graph]
  by Program Dependency Graph (PDG)
  (Look at Definition 3.1 in Flexeme paper)
\end{Definition}

\begin{Definition}[Multi-version Program Dependence Graph]
(Look at Definition 3.2 in Flexeme paper)
\end{Definition}

\begin{figure*}[t]
	\centering
        \includegraphics[width=5.3in]{figures/multi-version-graph.png}
        \vspace{-6pt}
	\caption{FIXME: Multi-Version PDG}
	\label{fig:pdg}
\end{figure*}

\begin{Definition}[Changed/Un-changed Statements]
Label 1,2: unchanged, label 1: deleted,...
\end{Definition}


\begin{Definition}[Context]
The context of a changed node is ...
\end{Definition}

\subsection{Architecture Overview}


\section{Context-aware, Graph-based, Code Change, Clustering Learning Model}
\label{clustering-model:sec}

Let us explain in details our approach. While the first step (building
the multi-version {\mvpdg}) is presented in
Section~\ref{overview:sec}, we present in this section our
context-aware, graph-based, code change (CC) clustering learning
model.  This step has two tasks: 1) Taking the computed {\mvpdg} to
learn the code change representation vectors (embeddings), and 2)
performing clustering on those embeddings to cluster the changed
statements.  During training, we have the ground truth on the clusters
of the changed statements, thus, we have the cluster labels (concern
1, concern 2, etc.) for the changed nodes in {\mvpdg}. During
predicting (i.e., clustering), the input {\mvpdg} will be fed into the
trained clustering learning model to produce the cluster labels for
the changed nodes/statements. For the given {\mvpdg}, the number of
clusters is unknown to the model, and it decides that number via its
agglomerative clustering. Let us detail the two tasks of our CC
clustering learning model.

%traing and predicting


\subsection{Context-aware, Code Change Representation Learning}
\label{vector:sec}

\begin{figure*}[t]
	\centering \includegraphics[width=5.8in]{figures/STEP_2-new.png}
	\vspace{-6pt}
	\caption{Context-aware, Graph-based, Code Change, Clustering Learning Model}
	\label{fig:step-2}
\end{figure*}

The goal of this task is to build the vector representations (i.e.,
embeddings) for the code changes (i.e., the changed statements)
represented by the changed nodes in {\mvpdg}. A characteristic of the
embeddings for the code changes is {\em context-aware} or {\em
  contextualized}. That is, the same code change in different contexts
will have different embeddings.

\subsubsection{{\bf Label, Graph-based Convolutional Network (Label-GCN)}}

To achieve that, {\tool} first feeds the multi-version {\mvpdg} to a
graph-based, deep learning model to learn the contextualized
embeddings for the nodes in the graph. Because {\mvpdg} contains the
labels (i.e., $i$, $j$, $(i,j)$) representing the (un)changes of the
nodes, to better learn the useful information from the node features
with the labels, {\tool} uses Label-GCN~\cite{} to model the graph.


%Before doing the clustering, in this step, \tool needs to get the representation vector $v'_c$ for each changed node $n_c$ first. To achieve it, \tool firstly put the {\mvpdg} into a graph-based deep learning model to learn the representation vector for each node in the graph. Because the {\mvpdg} contains the labels for each node, to better learn the useful information from the node features with the labels, \tool uses the Lable-GCN \cite{} model to do so.

%In the Label-GCN, similarly to the normal GCN \cite{}, it takes the graph with the node features as the input and generates the representation vector for each node as the output. Compared with normal GCN, it also accepts the node labels in the first layer. In this case, when dealing with a central node, the model is allowed to see the labels of the neighbors, and the labels can then become part of the feature vector. With this, Label-GCN calculates the representation vectors in the first layer as follow:

Simlar to GCN~\cite{yi}, Label-GCN~\cite{yi} takes the graph with the
node features as the input and produce the vectors for the nodes, when
considering the features of the neighboring nodes of each node.  In
addition, the Label-GCN also accepts the node labels in the first
layer. For the node under consideration, it can take the labels of the
neighboring nodes in account as part of the feature vectors. With
that, it computes the vectors in the first layer as follows:
\begin{equation}\label{eq1}
	H^1 = \sigma [(\hat{A}X-diag(\hat{A})\sum_{j=1}^{K}e_je^T_j)W^0]
\end{equation}
\begin{equation}\label{eq2}
	\hat{A} = \tilde{D}^{-\frac{1}{2}}\tilde{A}\tilde{D}^{-\frac{1}{2}}
\end{equation}
\begin{equation}\label{eq3}
	\tilde{A} = A + I
\end{equation}
Where $H$ is the output for the first hidden layer; $A$ is the
adjacency matrix; $\tilde{D}$ is the diagonal node degree matrix; $W$
is the weight matrix; $X$ is the input and $X \in R^{nx(d+K)}$; $n$ is
the number of nodes; $d$ is the dimension of node features; $K$ is the
number of types of node labels in the input; $e_je^T_j$ a single-entry
matrix; and $-diag(\hat{A})\sum_{j=1}^{K}e_je^T_j$ is used to
eliminates the self-loops for the components of the feature vectors
corresponding to the labels.

%After the first layer, the result layers in Label-GCN follow the same process as normal GCN to calculate the hidden status. The formula of the following layers is as follow:

In Label-GCN, the following layers after the first one follows the
same process as in the GCN model~\cite{yi} to compute the hidden
states. The computation in a following layer $l$ is as follows:
\begin{equation}\label{eq4}
	H^{l+1} = \sigma (\hat{A}H^lW^l), l \geq 1 
\end{equation}

\subsubsection{{\bf Using Label-GCN to model {\mvpdg}}}
\label{sec:preprocess}
This section explains how we process the multi-version {\mvpdg} to
produce the input for the Label-GCN model. For each node $n$ of a
statement $s$ in {\mvpdg}, we break it down into the code tokens $t$,
and build the word embeddings $e_t$ via Glove~\cite{yi}. The vector
for the node $n$ is the average vector $Avg_n$ of the vectors of all
the tokens $t$ within the corresponding statement $s$. Because each
node $n$ has an (un)-changed label $i$, $j$, or $(i,j)$ for the
versions, we combine $Avg_n$ with the one-hot vector of the length 3
representing the labels $i$, $j$, or $(i,j)$. As a result, the
combined vector $v_n$ (with the length of $len(Avg_n)$+3) is the node
feature vector for $n$. Then, we build the graph with the same
structure as {\mvpdg} in which a node $n$ is replaced with the node
vector $v_n$, and feed that graph to the Labe-GCN model to obtain the
embeddings $V_n$ for all the nodes in {\mvpdg}.

%In \tool, we firstly use GloVe \cite{} to learn the word embedding $e_t$ for each token $t$ in each statement $n$ in the source code. Because in {\mvpdg}, each node is a statement $n$, we calculate the average embedding $Avg_n$ for each node based on the word embedding $e_t$ for each token inside of the statement. By combining $Avg_n$ with the known label $i$, $j$, and $i, j$ as one-hot labels, node embedding length become $len(Avg_n) + 3$ and \tool regards the combined vector $Avg'_n$ as the node feature vector for node $n$. Then, we put the {\mvpdg} with the node feature vectors $Avg'_n$ into the Label-GCN model to get the representation vector $v$ for each node $n$ in the graph.

\subsubsection{Building Contexts and Contextualized Embeddings}
After using Label-GCN, we obtain the vectors $V_n$ for all the nodes
$n$. For a changed node $n_c$, we collect the nodes in its context
$ctx$ (i.e., all un-changed nodes that are the $k$-hop neighbor of $n$
together with all the inducing edges among them. We merge all the
vectors for the nodes in $ctx$ into a matrix accordingly to the order
of the statements in source code. We then use a fully-connected layer
to learn the vector $v_{ctx}$ representing the context $ctx$ of the
changed node $n_c$.

Finally, we perform a cross-product between the vector $v_{ctx}$ and
the vector $V{n_c}$ for a changed node $n_c$, to produce the
contextualized embedding $V^{*}_{n_c}$ for $n_c$.

%After we have $v$ for each node $n$, for each changed node $n_c$ in $n$, we pick the context for node $n$ which are the un-changed nodes $n_{uc}$ in $k$-hops neighbors. By merging all representation vectors $v_{uc}$ in the context based on the original node orders as a matrix, \tool uses a fully-connected layer to learn the context representation vector $v_{ctx}$ for node $n_c$. In the end, to get the final representation vector $v'_c$ for the changed node $n_c$, we uses the cross-product to merge $v_c$ and $v_{ctx}$.

Figure~\ref{fig:step-2} illustrates the process of building the
contextualized embeddings for the code change example in
Figure~\ref{fig:multi-version-pdg}. $S_3$, ..., $S_7$ are the
statements at the corresponding lines. After building {\mvpdg},
{\tool} goes through the process described in
Section~\ref{sec:preprocess} (i.e., building token embeddings,
statement embeddings, and combining with the labels) to produce the
node vectors $v_n$ for all the nodes in {\mvpdg}. The Label-GCN model
takes the graph with the vectors $v_n$ to produce the graph with the
same structure in which each node is represented by the vector $V_n$
($V_3$, ..., $V_7$). Let us use the node for $V_6$ as an example.  The
context of 1-hop neighbors includes $V_3$, $V_4$, and $V_7$. By
merging those vectors as a matrix and passing through a fully
connected layer, we obtain the vector $v_{ctx}$ representing the 1-hop
context for $V_6$. The cross-product vector  $V^{*}_6$ = $v_{ctx}$ $\times$ $V_6$
is the contextualized embedding representing the changed statement
$S_6$.


%Figure \ref{fig:step-2} can be an example to show how this step works in the \tool. First of all, \tool has the {\mvpdg} which is also shown in figure \ref{fig:multi-version-pdg} as the input. The $S3-S7$ represents the statement in $line-3$ to $line-7$. Bypassing through the Label-GCN, \tool gets the same structured graph with representation vectors $V3-V7$ for each node $S3-S7$.

%Then, we pick the added node $S6$ as an example. The context includes the un-changed nodes $S3, S4, S7$, and the representation vectors for them are $V3, V4, V7$. By putting them as a matrix and passing through a fully connected layer, we have the context representation vector $v_{ctx}$ for $S6$.

%Then, we use the cross-product to calculate $v_{ctv} x V6$. The calculation result is the final code change representation vector $v'_{S6}$ for node $S6$.  

\subsection{Code Change Clustering Learning}
\label{clustering:sec}

After building the contextualized embeddings $V^{*}_{n_c}$ for all the
changed nodes in {\mvpdg}, {\tool} performs clustering on those
vectors to untangle the commit. We have modified the hierarchical
agglomerative clustering algorithm~\cite{yi} to make it a deep
learning model to cluster those vectors based on the clusters of the
changes in the training data. The training process for our CC
clustering learning model works  as follows.

{\em \underline{Step 1.}} We first consider each changed node $n_c$ as a
separate cluster $CL_c$.

{\em \underline{Step 2.}} We merge any two clusters whose cluster
similarity is the largest and higher than a threshold $T$.

{\em \underline{Step 3.}} We continue the merging to form larger
clusters until there is no cluster that can be merged. After this
step, we obtain the predicted clusters $CL$ at the current iteration.

{\em \underline{Step 4.}} The predicted clusters $CL$ at this iteration
are compared against the correct clusters $CL_{oracle}$ in the ground
truth. We develop a loss function for our training to minimize the
differences between the predicted clusters $CL$ and the correct
clusters $CL_{oracle}$. The parameters and the trainable threshold $T$
will be updated accordingly to the loss function for the next
iteration. The training will stop when the process converges and we
obtain the most suitable parameters for our learning model.

\subsubsection*{{\bf Cluster Similarity}} To compute the similarity between
two clusters $CL_1$ and $CL_2$, we take all the pairs of the changed
nodes $(n_1,n_2)$ where $n_1 \in CL_1$ and $n_2 \in CL_2$. We
then compute the cosine similarity between the corresponding vectors
$V^{*}_{n_1}$ and $V^{*}_{n_2}$ for each pair. The similarity between
two clusters is calculated as the average of all the similarity scores
of all the pairs $(n_1,n_2)$.

\subsubsection*{{\bf Trainable Threshold T}} In our CC clustering learning model,
we treat the merging threshold $T$ between smaller clusters as a {\em
  trainable parameter} of the model. $T$ is updated after each
iteration in accordance with the criteria defined in the loss function
as any other parameters in the model.

%After having the code change representation vector $v'_c$ for the changed node $n_c$ from the last step, \tool uses hierarchical agglomerative clustering algorithm on top of the $v'_c$ to do the clustering in this step. The hierarchical agglomerative clustering algorithm first regards each changed statement $n_c$ as a separated cluster $CL_c$. Secondly, it calculates the cluster similarity between every two clusters. Thirdly, \tool merges the two clusters that the cluster similarity between them is the largest one, and the cluster similarity is higher than the merging threshold $thres$. Finally, \tool repeats the small steps two and three until there are noclusters that can be merged, and then the left clusters $CL_{pre}$ are the final clustering results in this step.

%During this whole process, we set the merging threshold $thres$ as a trainable parameter, and during the training process, it can automatically get the most suitable value based on the dataset.

%And for the cluster similarity calculation between any two clusters $CL_{c1}$ and $CL_{c2}$, \tool firstly uses the cosine similarity to compute the similarity score $S_{c1,c2}$ between each two elements that are from $CL_{c1}$ and $CL_{c2}$. And then, the \tool calculates the cluster similarity by using the average linkage that is the average of all similarity scores $S_{c1,c2}$.

\subsubsection*{{\bf Loss Function}}

%Within this whole big step, when doing the training, we accept {\mvpdg} as input and ground true clustering results $CL_{label}$ as the training target to train the model. The loss function in training is used to minimize the differences between the predicted clustering results $CL_{pre}$ and $CL_{label}$ to get the most suitable parameters set for the deep learning model.

%When doing training, the ground true clustering results $CL_{label}$ and the predicted clustering results $CL_{pre}$ may have different number of clusters $Cluster_{label}$ and $Cluster_{pre}$. To make the model trainable, we transfer the clustering problem into a classification problem by picking the biggest number of clusters $C_{max} = max($ $Cluster_{label}, Cluster_{pre})$ as the number of classes. \tool uses the zero padding to fill the missed results for the clusters in $CL_{label}$ or $CL_{pre}$. Then \tool uses the cross entropy loss to train the model:

%\begin{equation}\label{new-func}
%	Loss^{*}(X, Y)= \min\limits_{\substack{X_n \in X\\ Y_m \in Y}}(-\sum_{\substack{x_i\in X_n\\ y_j 
%			\in Y_m}}W_ilog\frac{exp(x_i)}{exp(\sum\limits_{x_j \in X_n}x_j)}y_i)
%\end{equation}

%The above cross-entropy loss has been widely used in multi-class classification problems. However, the clustering problem is slightly different from the classification problem because the order of the clusters is not fixed as in the classification problem. So to address this problem, for each pair of predicted clustering results and ground true cluster results $\{x_1, ..., x_{C_{max}}\}$ and $\{y_1,...,y_{C_{max}}\}$, \tool could have $C_{max}!$ different possible orders for both predicted cluster results $X = \{X_1, ..., X_{C_{max}!}\}$ and ground true clustering results $Y = \{Y_1, ..., Y_{C_{max}!}\}$. 

We need to have a loss function that minimizes the differences between
the predicted set of clusters $CL$=$\{CL_1,CL_2, ...,$ $CL_M\}$ and
the correct set $CL_{oracle}$ = $\{Co_{1}, Co_{2},...,
Co_{N}\}$. Because the predicted and correct sets might have different
numbers of clusters ($M \neq N$), we first make them have the same
size $\mathcal{M}$ = $max(M,N)$ by adding the empty clusters to the smaller
set between $CL$ and $CL_{oracle}$.
%
Because we do not know what predicted cluster $CL_i$ in $CL$ is mapped
to a cluster $Co_j$ in the correct set $C_{oracle}$, we consider all
possible orders of the clusters in both $CL$ and $CL_{oracle}$, and
all possible maps between each cluster $CL_i$ and $Co_j$. Note that
the number of clusters is usually small ($\leq$ 6 as reported
in~\cite{nguyen-issre13}), thus, it is manageable to consider all
$\mathcal{M}!$ possible orders in $CL$ and all $\mathcal{M}!$ possible
orders in $C_{oracle}$.

Let us consider an order in $CL$ = $\{CL'_1,..., CL'_{\mathcal{M}}\}$,
and an order in $C_{oracle}$ = $\{Co'_{1},..., Co'_{\mathcal{M}}\}$.
For a changed statement $s_c$ with the corresponding changed node
$n_c$ in {\mvpdg}, we build the 1-hot vector $X$ that represents the
cluster for $n_c$ predicted by the model as follows. If $n_c$ is
predicted to belong to a cluster $CL'_{i}$, the value at the $i$
position of the vector $X$ will be set to 1, otherwise it is set to
0. For example, if $\mathcal{M}$=4, the changed node $n_c$ is
predicted to belong to the cluster $CL'_3$, the vector
$X$=$\{0,0,1,0\}$. A changed statement can belong to multiple
clusters. Similarly, we build the 1-hot vector $Y$ to represent the
correct cluster for $n_c$ in the oracle: if $n_c$ belongs to a cluster
$Co'_{i}$, the value at the $i$ position of the vector $Y$ will be set
to 1, otherwise it is 0.

For a specific order in $CL$ = $\{CL'_1,..., CL'_{\mathcal{M}}\}$, and
a specific order in $C_{oracle}$ = $\{Co'_{1},...,
Co'_{\mathcal{M}}\}$, for a changed node $n_c$, we use the
cross-entropy loss function in the multi-class classification problem as
follows: ($X$=$\{x_1,...,x_{\mathcal{M}}\}$, $Y$=$\{y_1,...,y_{\mathcal{M}}\}$)
\begin{equation}\label{loss-func}
	Loss(X,Y) = -\sum^{\mathcal{M}}_{i=1}W_ilog\frac{exp(x_i)}{exp(\sum^{\mathcal{M}}_{j=1}x_j)}y_i
\end{equation}
To adjust Formula~\ref{loss-func} for our clustering problem, we need
to consider all possible orders in the cluster set $CL$ and those in
$C_{oracle}$. Thus, the cross-entropy loss function
for a changed node $n_c$ of a changed statement $s_c$ is the minimum
value among all the values on the right-hand side of
Formula~\ref{loss-func}. Thus, our loss function is as follows.
\begin{equation}\label{eq6}
	Loss'(X, Y)= \min\limits_{\substack{all (\mathcal{M}!)^{2}\\orders}}(-\sum^{\mathcal{M}}_{i=1}W_ilog\frac{exp(x_i)}{exp(\sum^{\mathcal{M}}_{j=1}x_j)}y_i)
\end{equation}
%The final loss function is the total loss function for all the pairs of the
%cluster $X$ in $CL$ and its corresponding $Y$ in $CL_{oracle}$, and
The loss function for each changed node $n_c$ will be used for
the model to adjust the parameters in the next iteration.

%\begin{equation}\label{eq6}
%	Loss'(X, Y)= \min\limits_{\substack{X_n \in X\\ Y_m \in Y}}(-\sum_{\substack{x_i\in X_n\\ y_j 
%			\in Y_m}}W_ilog\frac{exp(x_i)}{exp(\sum\limits_{x_j \in X_n}x_j)}y_i)
%\end{equation}

\subsubsection*{Predicting/Clustering Process}
After training, we obtain all model's parameters and the trainable
merging threshold. For clustering, {\tool} takes a {\mvpdg} as input
and generates the clusters $CL$.


\section{Updating Clusters via Code Clone Detection}

\section{Empirical Evaluation}
\label{eval:sec}

\subsection{Research Questions}

To evaluate {\tool}, we seek to answer the following questions:

\noindent\textbf{RQ1. Comparative Study on C\# Dataset.}  How well does {\tool} perform in comparison with the state-of-the-art untangling approaches on C\# dataset?

\noindent\textbf{RQ2. Comparative study on Java Dataset.}  How well does {\tool} perform in comparison with the state-of-the-art untangling approaches on Java dataset?

\noindent\textbf{RQ3. Within Project Analysis.}
How well does {\tool} perform when doing training and testing in the same project.

\noindent\textbf{RQ3. Overlapping Analysis.}
What are the differences between the tangled commits that the state-of-the-art baseline can deal with and the tangled commits that {\tool} can deal with.

\noindent\textbf{RQ5. Sensitivity Analysis.} How do the key features affect the overall performance of {\tool}?

\noindent\textbf{RQ6. Case Study.}  How well does {\tool} perform in the real world cases?



\subsection{Datasets}

\begin{table}[t]
	\caption{C\# Dataset Overview}
	\vspace{-0.1in}
	\begin{center}
		\footnotesize
		\tabcolsep 4pt
		\renewcommand{\arraystretch}{1} \begin{tabular}{p{3cm}<{\centering}|p{0.8cm}<{\centering}p{0.8cm}<{\centering}p{0.8cm}<{\centering}}
			
			\hline
			\multirow{2}{*}{Project}                  & \multicolumn{3}{c}{Concerns}\\
			\cline{2-4}
			                     & 2 & 3& Overall\\
			\hline
			
			Commandline        &  308 & 32  &   340        \\
			CommonMark        &  52 & 0  &   52        \\
			Hangfire        &  229 & 87  &   316        \\
			Humanizer        &  85 & 4  &   89        \\
			Lean        &  154 & 24  &   178        \\
			Nancy        &  284 & 67  &   351        \\
			Newtonsoft.Json        &  84 & 7  &   91        \\
			Ninject        &  82 & 0  &  82        \\
			RestSharp        &  95 & 18  &   113        \\
			\hline
			Overall        &  1373 & 239  &  1612        \\
			\hline
		\end{tabular}
		\label{C-dataset}
	\end{center}
\end{table}




We have conducted our evaluation on two datasets. The first one is the
C\# dataset that has been used in a prior commit-untangling research,
Flexeme~\cite{flexeme-fse20}. This C\# dataset contains 1,612 tangled
commits created from 21k commits in 9 open-source projects. We use
this C\# dataset for all RQs except RQ2. To evaluate {\tool} on Java
in RQ2, we collected a new dataset by following the same procedure
from Shen {\em et al.}~\cite{smartcommit-fse21} since we aim to
compare {\tool} against their tool,
SmartCommit~\cite{smartcommit-fse21}. This Java dataset contains XXX
tangled commits created from XXk commits in XX open-source projects.

%We conduct our study on two datasets. One is for programming language C\# while the other is for Java's programming language. The C\# dataset is from the existing study from Partachi et al. \cite{flexeme-fse20}. The whole dataset contains 1612 tangled commits that are created from 21616 commits in 9 open source projects. This dataset has been used to do the evaluation for RQ1, RQ3, RQ4, and RQ5. The Java dataset is collected by us following exactly the same procedure in Shen et al.'s \cite{smartcommit-fse21} study. The dataset contains XX tangled commits that are created from XX commits in XX open source projects. This dataset is used to do the evaluation in RQ2.


\subsection{Experimental Methodology}

\noindent\textbf{RQ1. Comparison on Method-Level DL-based VD Approaches.}

\textit{\underline{Baselines.}} We compare {\tool} with the state-of-the-art untangling approaches that are worked on C\# dataset in this RQ.

\begin{itemize}
	\item Barnett et al. \cite{barnett2015helping}: An automatic technique for decomposing changesets and evaluate its effectiveness through both a quantitative analysis and a qualitative user study
	\item Herzig et al. \cite{herzig2016impact}: An automatic approach with multi-predictors to untangle code changes.
	\item $\sigma-$PDG + CV \cite{flexeme-fse20}: The approach that uses multi-version PDG to do the clustering to untangle the commits.
	\item Flexeme \cite{flexeme-fse20}: An approach that builds a new defined name flow graph from commits, then applies 	agglomerative alustering using graph similarity to that newly built graph to untangle its commits.
\end{itemize}

\textit{\underline{Procedure.}}
In this RQ, we are focusing on the baselines that are workable on the C\# dataset. Within the C\# dataset, we firstly order all tangled commits in the dataset based on the latest modified time in the commit log from oldest to the newest. And then we split all of the tangled commits into 80\%, 10\%, and 10\% to be used for training, tuning, and testing for \tool, respectively. We try to use the existing tangled commits to let model learn the features and then use the newest tangled commits to test the performance of the model. As for the baselines, because they don't need model training, we only evaluate the model performance on the 10\% testing dataset for fair comparison. We use AutoML~\cite{NNI} on \tool to automatically tune hyper-parameters on the tuning dataset.

\noindent\textbf{RQ2. Comparison with other Interpretation Models for Fine-grained Interpretation.}

\textit{\underline{Baselines.}} We compare {\tool} with the state-of-the-art untangling approaches that are worked on Java dataset in this RQ.

\begin{itemize}
	\item Base-1 \cite{smartcommit-fse21}: The rule-based approach that putting all changes into one group. 
	\item Base-2 \cite{smartcommit-fse21}: The rule-based approach that putting changes in each file into one group 
	\item Base-3 \cite{smartcommit-fse21}: The rule-based approach that considering only def-use, use-use and	same-enclosing-method relations.
	\item SmartCommit \cite{smartcommit-fse21}: A graph-partitioning-based interactive approach to tangled changeset decomposition that leverages the efficiency of algorithms and the knowledge of developers.
\end{itemize}

\textit{ \underline{Procedure}.}
In this RQ, we following the similar procedure in RQ1 to order all tangled commits based on the last modified time from oldest to the newest. And then we split the dataset into 80\%, 10\%, and 10\% to be used for training, tuning, and testing for \tool, respectively. As for all baselines in this RQ, they all don't need training dataset, so we directly evaluate the performance of them on the 10\% testing dataset. The same as RQ1, we use AutoML~\cite{NNI} on \tool to automatically tune hyper-parameters on the tuning dataset.

\noindent\textbf{RQ3. Within Project Analysis.}

We used the C\# dataset in RQ1 as the dataset that used to do the analysis in this RQ. For each project in the dataset, we sorted the tangled commits based on the time in the commit log. Then we split the commits in each project into 80\%, 10\%, and 10\% for training, tuning, and testing. The oldest data is used to train the model while the newest data is used to test the model performance. We separately train, fine-tune, and test the model on each project in the dataset. But because the size of the dataset that the model uses each time is very small and limited after separating the data into each project, based on our effort on trying to run the model on each project, only on the project $Commandline$, the model could provide meaningful results with acceptable influence caused by over-fitting or under-fitting. So in the result section of this RQ, the results are reported only on the project $Commandline$. 

\noindent\textbf{RQ4. Overlapping Analysis.}

{\color{red}{Need to add more.}}

\noindent\textbf{RQ5. Sensitivity Analysis.}

We first use \tool as the base model. We then remove the context vector from the model to evaluate the impact of context information. We also build the other model by removing the code clone from the base model to evaluate the impact of code clone technique. We used the C\# dataset and the same experiment setting as in RQ1.

\noindent\textbf{RQ6. Case Study.} 

We use real world cases to analysis the performance of \tool.

\noindent\textbf{Evaluation Metrics}


In all experiments, we measure the performance of all approaches with two evaluation metrics. The main metric we are using is the untangling accuracy $Accuracy^{(1)}$. It is defined as follow:

\begin{equation}\label{eq7}
	Accuracy^{(1)} = \frac{\#\:of\:Correctly\:Clustered\:Changed\:Statement}{\#\:of\:Changed\:Statement\:in\:the\:Commit}
\end{equation}

In RQ1, because we directly use the dataset from Partachi et al. \cite{flexeme-fse20} and compare with this study as a baseline in this research question, we also use the metric that has been used in Partachi et al.'s \cite{flexeme-fse20} study for doing the evaluation to fully analysis the differences on the performance. The accuracy metric in Partachi et al.'s \cite{flexeme-fse20} study is defined as follow:

\begin{equation}\label{eq8}
	Accuracy^{(2)} = \frac{\#\:of\:Correctly\: Clustered\: Changed\: Statement}{\#\: of\: Nodes\: in\: the\: Graph}
\end{equation}
\section{Experimental Results}
\label{result:sec}

\subsection{RQ1. Comparative Study on C\# Dataset}
\label{rq1:sec}

\begin{table}[t]
	\caption{RQ1. Comparison Results on C\# Dataset}
	\vspace{-0.1in}
	\begin{center}
		\footnotesize
		\tabcolsep 4pt
		\renewcommand{\arraystretch}{1} \begin{tabular}{p{1.4cm}<{\centering}|p{1.3cm}<{\centering}p{1.2cm}<{\centering}p{1.4cm}<{\centering}p{0.8cm}<{\centering}|p{0.7cm}<{\centering}}
			
			\hline
			Approach          & Barnett {\em et al.} & Herzig {\em et al.} & $\delta-$PDG + CV & Flexeme & {\tool}\\
			\hline
			$Accuracy^{(c)}$   &        0.08    &		0.29	&		0.35      & 0.33	& 0.45     \\
			$Accuracy^{(a)}$   &        0.12    &		0.70	&		0.81	  & 0.83    & 0.89      \\
			\hline
		\end{tabular}
		\label{RQ1-result}
	\end{center}
\end{table}

\textcolor{red}{Yi: consider adding Table 2 in Flexeme paper to our paper on the statistics of the data}.

\textcolor{red}{Yi: can you extend this Table 1 into similar to Table 3 in Flexeme paper to include all projects and different numbers of concerns/clusters.}

\textcolor{red}{Yi: add Figure 4(a) in Flexeme paper to our paper to show
  the accuracy distribution for different numbers of concerns.}

\tool can improve the $Accuracy^{(1))}$ by $462.5\%, 55.2\%, 28.6\%, $ and $36.4\%$ and can improve the  $Accuracy^{(2))}$ by $641.7\%, 27.1\%, 9.9\%, $ and $7.3\%$ comparing with Barnett et al., Herzig et al., $\sigma-$PDG + CV, and Flexeme.


\subsection{RQ2. Comparative study on Java Dataset}

\begin{table}[t]
	\caption{RQ2. Comparison on Java Dataset ($Accuracy^a$\%)}
	\vspace{-0.1in}
	\begin{center}
		\footnotesize
		\tabcolsep 4pt
		\renewcommand{\arraystretch}{1} \begin{tabular}{p{0.2cm}<{\centering}|p{0.10cm}<{\centering}p{0.10cm}<{\centering}p{0.10cm}<{\centering}p{0.10cm}<{\centering}p{0.10cm}<{\centering}p{0.10cm}<{\centering}p{0.10cm}<{\centering}p{0.10cm}<{\centering}p{0.10cm}<{\centering}p{0.10cm}<{\centering}p{0.10cm}<{\centering}p{0.10cm}<{\centering}p{0.10cm}<{\centering}p{0.10cm}<{\centering}p{0.10cm}<{\centering}p{0.10cm}<{\centering}p{0.10cm}<{\centering}p{0.10cm}<{\centering}p{0.10cm}<{\centering}p{0.10cm}<{\centering}|p{0.10cm}<{\centering}}
			
			\hline
				 & NE & SB & ES & RJ & SF & GU & RE & DU & LA & GL & GH & ZX & JA & AR & BK & AP & PD & DR & FJ & EB & OA \\
			\hline
			B1   &  14 &  18  &  21  &  19  & 17   &  23  & 22   &  17  &  20  &  18  &  23  &  25  &  16  &  18  & *  &  19  & *  & 20   & 21   &  20  & 19  \\
			B2   & 29   &  23  &  24  &  28  & 24   & 31   & 26   &  19  &  20  &  25  &  27  &  29  &  21  & 24   & *  &  27  & *  & 29   & 28   &  22  & 26  \\
			B3   &  25  &  21  &  27  & 24   & 26   & 25   & 19   & 17   &  21  &  20  & 24   & 20   & 23   & 27   & *  & 22   & *  &  18  & 23   &  21  & 23  \\
			SC   & 36   & 29   & 34   & 31   & 32   & 33   & 29   & 24   & 31   & 30   & 33   & 35   & 26   & 28   & *  &  31  & *  & 32   & 34   &  27  & 31  \\
			\hline
			UT   &  35  & 31   & 36   & 33   & 29   & 37   & 34   & 28   & 32   & 30   &  35  & 38   & 25   &  27  & *  &  33  & *  &  34  &  36  &  31  & 33  \\
			\hline
		\end{tabular}
		\label{RQ2-result}
		B1: Base-1, B2: Base-2, B3: Base-3, SC: SmartCommit, UT: Utango, NE: netty, SB: spring-boot, ES: elasticsearch, RJ: RxJava, SF: spring-framework, GU:guava, RE: retrofit, DU: dubbo, LA: lottie-android, GL: glide, GH: ghidra, ZX: zxing, JA: jadx, AR: arthas, BK: butterknife, AP: apollo, PD: proxyee-down, DR: druid, FJ: fastjson, EB: EventBus, OA: Overall
	\end{center}
\end{table}



Training of \tool on $80\%$ of Java dataset takes XX hours and predicting on $10\%$ of Java dataset takes XX seconds for each commit.

\subsection{RQ3. Within-Project Analysis}

\begin{table}[t]
	\caption{RQ3. Within-Project Result}
	\vspace{-0.1in}
	\begin{center}
		\footnotesize
		\tabcolsep 4pt
		\renewcommand{\arraystretch}{1} \begin{tabular}{p{1.4cm}<{\centering}|p{3cm}<{\centering}|p{3cm}<{\centering}}
			
			\hline
			Approaches          & \tool (Within Project) & \tool (Cross Project)\\
			\hline
			$Accuracy^{(1)}$   &      0.47          &		0.45	       \\

			\hline
		\end{tabular}
		\label{RQ3-result}
	\end{center}
\end{table}


\subsection{{\bf RQ4. Overlapping Analysis.}}

\begin{table}[t]
	\caption{RQ4. Overlapping Analysis.}
	\vspace{-0.1in}
	\begin{center}
		\footnotesize
		\tabcolsep 4pt
		\renewcommand{\arraystretch}{1} \begin{tabular}{p{1.4cm}<{\centering}|p{1.3cm}<{\centering}p{1.2cm}<{\centering}p{1.4cm}<{\centering}p{0.8cm}<{\centering}|p{0.7cm}<{\centering}}
			
			\hline
			Approaches          & Barnett et al. & Herzig et al. & $\sigma-$PDG + CV& HEDDLE & \tool\\
			\hline
			\hline
		\end{tabular}
		\label{RQ4-result}
	\end{center}
\end{table}

\subsection{RQ5. Analysis on Code Change Embeddings}




%\subsection{RQ6. Time Complexity}




\input{sections/illustrations}



\section{Related Work}
\label{related:sec}

Tangled commits have been reported by researchers to have negative
impacts on software
maintenance~\cite{tao-fse12,kim-emse16,kim-msr13,hill-tse12,nguyen-issre13,flexeme-fse20,smartcommit-fse21}.
%They have caused negative impacts on software maintenance as
%explained~\cite{tao-fse12,flexeme-fse20,nguyen-issre13}.
%including hampering code comprehension~\cite{tao-fse12}, reducing the
%separation of concerns~\cite{flexeme-fse20}.  , and even reducing the
%accuracy of bug prediction models that rely on bug-fixing code
%commits for training.
%The automated approaches to untangle the commits can be divided
%into two categories: {\em mining software repositories} and {\em
%  program analysis}.

\vspace{1pt}
\noindent {\em Mining Software Repositories.} Herzig {\em et
  al.}~\cite{kim-msr13,kim-emse16} combine confidence voting with
agglomerative clustering.
%on the change operations.
Each confidence voter is responsible for an important aspect, e.g.,
call-graphs, change couplings, data dependencies, and distance
measures. In Kirinuki {\em et al.}'s~\cite{higo-apsec16, higo-icpc14},
if there is a commit $m$ including the same changes as a past commit
and other changes, the commit $m$ is called inclusive change and
considered as tangled. Dias {\em et al.}~\cite{dias-saner15} uses
confidence voting on fine-grained change events in an IDE and
partition them.
%
%Dias {\em et al.}~\cite{dias-saner15} focus on interactive IDE by
%using confidence voting on fine-grained change events. A clustering
%algorithm is used to partition the tangled changes.

%However, the voters are independent, thus, do not reflect well the
%interdependency nature of program elements under change. In contrast,
%Kirinuki {\em et al.}~\cite{higo-apsec16, higo-icpc14} rely on the
%histories of the co-changes to split the tangled code changes before
%they are committed. However, they do not consider the relations among
%the changes such as data or control dependencies. Dias {\em et
%  al.}~\cite{dias-saner15} also use confidence voters, but on the
%fine-grained change events in an IDE. The scores are converted into
%the similarity ones via a Random Forest Regressor, which are used in
%the agglomerative clustering to partition the tangled changes.

\vspace{1pt}
\noindent {\em Static Analysis.} Roover {\em et al.}'s
approach~\cite{roover-scam18} builds PDG and computes the changes to
the ASTs of the files in a commit. It then groups these fine-grained
changes according to the slices through the PDG they belong to.
ClusterChanges~\cite{barnett-icse15} relates separate regions of
change within a changeset of a commit by using static analysis to
uncover relationships such as definitions and their uses present in
these regions. {\tool} adapts multi-version PDG from
Flexeme~\cite{flexeme-fse20}, however, we build the contextualized
embeddings for the code changes and a model to learn to cluster,
rather than clustering using graph similarity on multi-version PFG.
SmartCommit~\cite{smartcommit-fse21} uses a graph partition algorithm
on code changes related via several types of links, representing
different purposes.

There are a rich literature on supervised hierarchical
clustering~\cite{pmlr-v97-yadav19a,finley-icml05,liu13,GuhaIBB15,kenyon-dean-etal-2018-resolving,tie19}.
%A dissimilarity function is learned from the labeled data and used to
%partition unlabeled data~\cite{finley-icml05}.
The dissimilarity between cluster pairs~is measured via a linkage
function $F$~\cite{GuhaIBB15,pmlr-v97-yadav19a}. Learning
$F$ is performed by training the pairwise
dissimilarity function to predict dissimilarity for all within- and
across-cluster data
pairs~\cite{pmlr-v97-yadav19a,kenyon-dean-etal-2018-resolving}.
In {\tool}, supervised-learning clustering is made with the loss
function.
%defined on the classes in the classification.

%problem.

%Related work on embeddings for code changes ...


\section{Conclusion}
\label{conclusion:sec}

We present {\bf \tool}, a machine learning (ML)-based approach that
learns to untangle the changes in a commit. We develop a {\em novel
  code change clustering learning model} that learns to cluster the
code changes, represented by the embeddings, into different groups
with different concerns. {\tool} overcomes the key issue with the
static analysis approaches in which the boundaries across concerns in
a commit do not necessarily and naturally map to clustering criteria
of a clustering algorithm.  Our ML direction fits well with this
problem thanks to a methodology from Herzig {\em et
  al.}~\cite{kim-msr13} and Flexeme~\cite{flexeme-fse20}, to collect
the changes in the same concern and merge them to build the tangled
commits to form a training dataset.


\newpage

\balance

%\bibliographystyle{plain}
%\bibliographystyle{ACM-Reference-Format}
\bibliographystyle{ACM-Reference-Format}

\bibliography{References}

\end{document}
