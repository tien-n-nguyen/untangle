\subsection{{\bf RQ3. Within-Project Analysis}}




Table~\ref{RQ3-result} shows {\tool}'s results
%comparison between running {\tool}
in the within and cross-project settings. As seen, with sufficient
within-project data, $Accuracy^{c}$ in the within-project setting is
slightly better than that of the cross-project setting for
both C\# and Java projects.
%The trend is consistent for the commits with 2 or 3 concerns.
This is expected since the changes in the same project could be more
similar than the changes across projects.~Thus, {\tool} works well in
both within- and cross-project settings for C\# and Java projects.


%As we mentioned in the methodology section, because only on the project $Commandline$, \tool can have meaningful results, in Table~\ref{RQ3-result}, all results are only from the project $Commandline$ for a fair comparison. As seen, for $2$ concerns data, $3$ concerns data, and the overall performance, the within project setting can have $2.2\%, 4.3\%,$ and $2.2\%$ higher $Accuracy^c$ compared with the cross-project setting. It shows that using the data from the same project can help the model learn the feature more accurately. However, because the improvement percentages are all less than $5\%$, it proves that \tool can have consistent and similar performance on both the within and cross-project settings.


