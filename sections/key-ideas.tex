%\vspace{-12pt}
\subsection{Key Ideas}
\label{ideas:sec}

Inspired from the above observations, we propose {\tool} with the
following ideas.

%\subsubsection{\bf Key Idea 1 [Code Change Clustering Learning Model]}

\subsubsection{Key Idea 1 [Code Change Clustering Learning Model]}

Instead of deciding a deterministic clustering criterion on the
concrete artifacts (PDGs, program slices, code changes, operations, or
change graphs), from Observation 1, we build an ML model to untangle
the commits by learning to cluster code~changes represented by
embeddings, w.r.t. different concerns. The model learns from
the~history of the co-changed statements in the same commits for the
same concerns, and applies to cluster the changes in the current
commit.  We also modify an agglomerative clustering algorithm into
a~super\-vised-learning clustering model via trainable parameters and
a loss function to compare the predicted and correct clusters.

%\subsubsection{\bf Key Idea 2 [Context-aware, Graph-based
%    Representation Learning for Code Changes]}

\subsubsection{Key Idea 2 [Context-aware, Graph-based
    Representation Learning for Code Changes]}

We design a context-aware, graph-based, representation learning model
to {\em learn the contextualized embeddings for the code changes} that
integrates {\em program dependencies} among the program elements, and
the {\em contexts} of~code changes. Leveraging the changes in the same
concern in the history, we train a Label, Graph-based Convolution
Network~\cite{label-gcn} to learn~the embeddings and learn to cluster
them.
%from the history of changed code for the same concerns.
For prediction, we build a supervised-learning agglomerative
clustering algorithm that operates on the embeddings of code changes
to produce the clusters for the purpose of untangling the commit. We
expect that supervised-learning clustering on vectors is more
effective than on those artifacts since the embeddings capture~richer
information integrated from dependencies and contexts.

%\subsubsection{\bf Key Idea 3 [Explicit Context Representation as a Weight to
%    Compute Vectors for Code Changes]}


\subsubsection{Key Idea 3 [Explicit Context Representation as a Weight to
    Compute Vectors for Code Changes]}

As seen in Observation 2, the contexts of the code changes can help
distinguish their concerns in the commits. We represent code changes
and the surrounding context of a change via the multi-version program
dependence~graph, $\delta$-PDG~\cite{flexeme-fse20}, consisting of the
elements of both versions before and after the changes, and program
dependencies. The context is defined as the surrounding nodes of the
changed statement node~in~that graph. The Label-GCN is used to model
the statements and their program dependencies in $\delta$-PDG as well
as to learn the vector representing the context for a change. The
context vectors are then used as the weights in learning
contextualized embeddings for the code changes.

%\subsubsection{\bf Key Idea 4 [Implicit Dependencies among Cloned Code]}

\subsubsection{Key Idea 4 [Implicit Dependencies among Cloned Code]}

As seen in Observation 3, the cloned code exhibits implicit
dependencies with regard to whether they can be changed in the same
commits for the same concerns. Thus, during the process of producing
the final result, we also integrate the code clone relationships to
adjust the clusters produced by the clustering learning model.

%Thus, the GCN model can learn the embeddings of the code changes with
%the consideration of the cloned code.
