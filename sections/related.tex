\section{Related Work}
\label{related:sec}

Tangled commits have been reported by several
researchers~\cite{tao-fse12,kim-emse16,kim-msr13,hill-tse12,nguyen-issre13,flexeme-fse20,smartcommit-fse21}. They
have caused negative impacts including hampering code
comprehension~\cite{tao-fse12}, reducing the separation of
concerns~\cite{flexeme-fse20}, and even reducing the accuracy of bug
prediction models that rely on bug-fixing code commits for training.
The automated approaches for untangling the commits can be divided
into two categories: {\em mining software repositories}, and {\em
  program analysis}.

\noindent {\bf Mining Software Repositories.} Herzig {\em et
  al.}~\cite{kim-msr13,kim-emse16} combine confidence voting with
agglomerative clustering on the change operations. Each confidence
voter is responsible for an important aspect including call-graphs,
change couplings, data dependencies, and distance measures. In
Kirinuki {\em et al.}'s approach~\cite{higo-apsec16, higo-icpc14}, if
there is a commit $m$ including the same changes as a past commit and
other changes, the commit $m$ is called inclusive change and
considered as tangled.  Dias {\em et al.}~\cite{dias-saner15} focus on
interactive IDE by using confidence voting on fine-grained change
events. A clustering algorithm is then used to partition the tangled
changes.

%However, the voters are independent, thus, do not reflect well the
%interdependency nature of program elements under change. In contrast,
%Kirinuki {\em et al.}~\cite{higo-apsec16, higo-icpc14} rely on the
%histories of the co-changes to split the tangled code changes before
%they are committed. However, they do not consider the relations among
%the changes such as data or control dependencies. Dias {\em et
%  al.}~\cite{dias-saner15} also use confidence voters, but on the
%fine-grained change events in an IDE. The scores are converted into
%the similarity ones via a Random Forest Regressor, which are used in
%the agglomerative clustering to partition the tangled changes.

\noindent {\bf Static Analysis.} Roover {\em et al.}'s
approach~\cite{roover-scam18} builds PDG and computes the changes to
the ASTs of the files in a commit. It then groups these fine-grained
changes according to the slices through the PDG they belong to.
ClusterChanges~\cite{barnett-icse15} relates separate regions of
change within a changeset of a commit by using static analysis to
uncover relationships such as definitions and their uses present in
these regions. {\tool} adapts multi-version PDG from
Flexeme~\cite{flexeme-fse20}, however, we build the contextualized
embeddings for the code changes and a model to learn to cluster,
rather than clustering using graph similarity on multi-version PFG.
SmartCommit~\cite{smartcommit-fse21} uses a graph partition algorithm
on code changes related via several types of links, representing
different purposes.

%check issre2013 for more related work


Related work on embeddings for code changes ...
