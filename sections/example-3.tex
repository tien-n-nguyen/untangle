\subsubsection{Example 3}

Figure~\ref{fig:motiv-context} shows another example in JHotDraw
project. At the commit r463, two changes at line 3 and line 13 are
exactly the same with the addition of \code{figure.isSelectable()} to
check whether a figure is selectable or not. However, those {\em two
  exact changes} occurred in two different methods \code{mousePressed}
and \code{updateHoverHandles} for two different concerns/purposes as
noted in the commit log. Line 4 was aimed to fix the tracking of a
dragging object as the mouse is pressed ({\em ``Fixed bug where
  setSelectable() on a dragging figure did not work''}).  Line 14 was
a fix for a different bug as the hovered object needs to be selected
({\em ``Selection classes now check the flag on the hovering figures
  to be selected.''}). The addition of \code{figure.isSelectable()} is
common for checking if a figure is selectable in the two tasks of
dragging and hovering a figure. Without the surrounding contexts,
one could not tell whether two changes are for the same purpose or not.

\vspace{3pt}
\noindent {\bf Observation 2 [Context].} {\em The surrounding context
  consisting of un-changed code is crucial to determine the
  concern of a change. The same change in two 
  contexts could be for different concerns}.

\vspace{3pt}
Figure~\ref{fig:motiv-cc} also gives us an interesting
observation. The changes to fix the NPE at lines 4, 7, and 10 are
exactly the same: \code{null} $\rightarrow$
\code{Collections.emptyList()}.  The three statements are cloned code
of one another to support for different stylesheets (author, user
agent, and inline stylesheets).  In fact, the changes in
Figure~\ref{fig:history} also occurred at the cloned statements. The
three changes belong to the same concern/purpose since they serve as a
fix for the same logic despite that there is no program dependency
among them. The existing untangling
approaches~\cite{flexeme-fse20,smartcommit-fse21,roover-scam18,barnett-icse15}
that require the explicit program dependencies among the changes with
the same purpose will not classify these committed code as belonging
to the same concern.

\vspace{2pt}
\noindent {\bf Observation 3 [Implicit Dependencies].} {\em Two
  fragments of cloned code have the same/similar logic, thus, have
  implicit dependencies, and could be modified in the same manner to
  serve the same purpose.}
