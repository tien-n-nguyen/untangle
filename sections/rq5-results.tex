\subsection{RQ5. Analysis on Code Change Embeddings}

To study the projections of the changed statements in the same and
different concerns, we first randomly chose the commits with two or
more clusters/concerns and in one of the cluster/concern, there are at
least two changed statements. Let us use $C$ to denote that
cluster/concern. We randomly chose two changed statements $S_1$ and
$S_2$ in $C$. We then randomly selected another changed statement
$S_3$ such that $S_3 \notin C$ and $S_3$ belongs to another cluster in
the same commit with $S_1$ and $S_2$. We measured the distance
$d_1(S_1,S_2)$ and $d_2(S_1,S_3)$. We repeated the process for all the
commits satisfying the above conditions until to get 384 triples of
($S_1, S_2, S_3$). Based on the population in our dataset, the size of
384 samples gives the confidence level of 95\% and the confidence
interval of 5\%.
%
We used statistical $p$-value to confirm our hypothesis
$H_1: d_1(S_1,S_2)$ $\leq$ $d_2(S_1,S_3)$. The null-hypothesis is
\textit{\textbf{$H_0: d_1(S_1,S_2) > d_2(S_1,S_3)$}}.

When we set the significance level $\alpha = 0.05$, the $p$-value is
$0.03$ (calculated on these 384 samples). In this case, the $p$-value
is smaller than $\alpha$, meaning the null hypothesis would be
rejected at the $\alpha = 0.05$ level. Therefore, our hypothesis
$H_1$: $d_1(S_1,S_2)$$ \leq$$ d_2(S_1,S_3)$ is confirmed.  That is,
the changed statements in the same concerns are projected nearer to
one another than the changed statements in different concerns. This
result is an indication that our {\em context-aware, graph-based embeddings
for code changes are of high quality and helpful in the code change
clustering into different concerns}.

%\textit{\textbf{$H_1: d_1(S_1,S_2) \leq d_2(S_1,S_3)$}}

%Moreover, we also reported the examples where the same changed
%statements in different concerns are projected farther away in the
%vector space.


%To better understand the quality of the code change embeddings, in this RQ, we selected 384 triples of ($S_1, S_2, S_3$) as the statistical analysis sample based on the confidence level of $95\%$ and the confidence interval $5\%$. Our hypothesis for the quality of code change embedding is that $d_1(S_1,S_2)$ $\leq$ $d_2(S_1,S_3)$. 

% which means that our code change embeddings are in good quality.


%\begin{figure}[t]
%	\centering
%	\lstset{
%		numbers=left,
%		numberstyle= \tiny,
%		keywordstyle= \color{blue!70},
%		commentstyle= \color{red!50!green!50!blue!50},
%		frame=shadowbox,
%		rulesepcolor= \color{red!20!green!20!blue!20} ,
%		xleftmargin=1.5em,xrightmargin=0em, aboveskip=1em,
%		framexleftmargin=1.5em,
%		numbersep= 5pt,
%		language=Java,
%		basicstyle=\scriptsize\ttfamily,
%		numberstyle=\scriptsize\ttfamily,
%		emphstyle=\bfseries,
%		moredelim=**[is][\color{red}]{@}{@},
%		escapeinside= {(*@}{@*)}
%	}
%	\begin{lstlisting}[]
%//-----------------------------------Concern-1-------------------------------
%    private void Dispose(bool disposing)
%    {
%(*@{\color{red}{-     \space \space\space\space\space\space\space\space      if% (this.disposed)}@*)
%(*@{\color{cyan}{+    \space\space\space\space\space\space\space\space\space   %     if (disposed)}@*)
%        {
%            return;
%        }
%        ...
%    }
%
%//-----------------------------------Concern-2-------------------------------
%    private HelpText AddOption(string requiredWord, int maxLength, OptionSpecification option, int widthOfHelpText)
%    {
%        ...
%        if (option.ShortName.Length > 0)
%        {
%(*@{\color{red}{-\space\space\space\space\space\space\space\space\space\space  %      if (this.addDashesToOption)}@*)
%(*@{\color{cyan}{+\space\space\space\space\space\space\space\space\space\space %      if (addDashesToOption)}@*)
%	    	{
%			    optionName.Append('-');
%	    	}
%    	...
%    }
%    
%    internal HelpText AddToHelpText(HelpText helpText, bool before)
%    {
%        return before
%(*@{\color{red}{- \space\space\space\space\space\space\space\space               ? this.AddToHelpText(helpText, line => helpText.AddPreOptionsLine(line)) : this.AddToHelpText(helpText, line => helpText.AddPostOptionsLine(line));}@*)
%(*@{\color{cyan}{+  \space\space\space\space\space\space\space\space 	? AddToHelpText(helpText, helpText.AddPreOptionsLine) : AddToHelpText(helpText, helpText.AddPostOptionsLine);}@*)
%    }

%	\end{lstlisting}
%	\vspace{-15pt}
%	\caption{Example for RQ5}
%	\vspace{-6pt}
%	\label{RQ5-example}
%\end{figure}
			
%The code in Figure \ref{RQ5-example} shows an example that contains two concerns. The statement in $Line-4$ in concern-1 is very similar to the statement in $Line-18$ in concern-2. Following the statistical analysis procedure, we pick the statement in $Line-4$ as $S_3$, the statement in $Line-18$ as $S_2$, and the statement in $Line-29$ as $S_1$. Then we calculate the $d_1(S_1,S_2)$ and $d_2(S_1,S_3)$ by using \tool to generate the representation vectors for each changed statement. The results are $d_1(S_1,S_2) = 0.285$ and $d_2(S_1,S_3)=0.491$. This result proves that the similar changed statements in different concerns are projected farther away in the vector space when \tool generates the embedding vectors for the code change statements. It also shows that \tool learns the context information well to distinguish the differences between the similar code changes in a different context.

