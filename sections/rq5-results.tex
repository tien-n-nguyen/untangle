\subsection{RQ5. Analysis on Code Change Embeddings}

To better understand the quality of the code change embeddings, in this RQ, we selected 384 triples of ($S_1, S_2, S_3$) as the statistical analysis sample based on the confidence level of $95\%$ and the confidence interval $5\%$. Our hypothesis for the quality of code change embedding is that $d_1(S_1,S_2)$ $\leq$ $d_2(S_1,S_3)$. To do a null-hypothesis significance test to verify it, we have:

\textit{\textbf{$H_0: d_1(S_1,S_2) > d_2(S_1,S_3)$}} 

\textit{\textbf{$H_1: d_1(S_1,S_2) \leq d_2(S_1,S_3)$}}

When we set the significance level $\alpha = 0.05$, the $p$-value is $0.03$ that is calculated on these 384 samples. In this case, the $p$-value is smaller than $\alpha$, meaning the null hypothesis would be rejected at the $\alpha = 0.05$ level. So our hypothesis $d_1(S_1,S_2)$$ \leq$$ d_2(S_1,S_3)$ is confirmed which means that our code change embeddings are in good quality.


\begin{figure}[t]
	\centering
	\lstset{
		numbers=left,
		numberstyle= \tiny,
		keywordstyle= \color{blue!70},
		commentstyle= \color{red!50!green!50!blue!50},
		frame=shadowbox,
		rulesepcolor= \color{red!20!green!20!blue!20} ,
		xleftmargin=1.5em,xrightmargin=0em, aboveskip=1em,
		framexleftmargin=1.5em,
		numbersep= 5pt,
		language=Java,
		basicstyle=\scriptsize\ttfamily,
		numberstyle=\scriptsize\ttfamily,
		emphstyle=\bfseries,
		moredelim=**[is][\color{red}]{@}{@},
		escapeinside= {(*@}{@*)}
	}
	\begin{lstlisting}[]
//-----------------------------------Concern-1-------------------------------
    private void Dispose(bool disposing)
    {
(*@{\color{red}{-     \space \space\space\space\space\space\space\space      if (this.disposed)}@*)
(*@{\color{cyan}{+    \space\space\space\space\space\space\space\space\space        if (disposed)}@*)
        {
            return;
        }
        ...
    }

//-----------------------------------Concern-2-------------------------------
    private HelpText AddOption(string requiredWord, int maxLength, OptionSpecification option, int widthOfHelpText)
    {
        ...
        if (option.ShortName.Length > 0)
        {
(*@{\color{red}{-\space\space\space\space\space\space\space\space\space\space        if (this.addDashesToOption)}@*)
(*@{\color{cyan}{+\space\space\space\space\space\space\space\space\space\space       if (addDashesToOption)}@*)
	    	{
			    optionName.Append('-');
	    	}
    	...
    }
    
    internal HelpText AddToHelpText(HelpText helpText, bool before)
    {
        return before
(*@{\color{red}{- \space\space\space\space\space\space\space\space               ? this.AddToHelpText(helpText, line => helpText.AddPreOptionsLine(line)) : this.AddToHelpText(helpText, line => helpText.AddPostOptionsLine(line));}@*)
(*@{\color{cyan}{+  \space\space\space\space\space\space\space\space 	? AddToHelpText(helpText, helpText.AddPreOptionsLine) : AddToHelpText(helpText, helpText.AddPostOptionsLine);}@*)
    }

	\end{lstlisting}
	\vspace{-15pt}
	\caption{Example for RQ5}
	\vspace{-6pt}
	\label{RQ5-example}
\end{figure}
			
The code in Figure \ref{RQ5-example} shows an example that contains two concerns. The statement in $Line-4$ in concern-1 is very similar to the statement in $Line-18$ in concern-2. Following the statistical analysis procedure, we pick the statement in $Line-4$ as $S_3$, the statement in $Line-18$ as $S_2$, and the statement in $Line-29$ as $S_1$. Then we calculate the $d_1(S_1,S_2)$ and $d_2(S_1,S_3)$ by using \tool to generate the representation vectors for each changed statement. The results are $d_1(S_1,S_2) = 0.285$ and $d_2(S_1,S_3)=0.491$. This result proves that the similar changed statements in different concerns are projected farther away in the vector space when \tool generates the embedding vectors for the code change statements. It also shows that \tool learns the context information well to distinguish the differences between the similar code changes in a different context.

