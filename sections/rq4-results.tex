\subsection{RQ4. Sensitivity Analysis}


\begin{table}[t]
	\caption{RQ4. Impacts of Key Features on Accuracy}
	\vspace{-0.1in}
	\begin{center}
		\footnotesize
		\tabcolsep 4pt
		\renewcommand{\arraystretch}{1} \begin{tabular}{p{3cm}<{\centering}|p{0.8cm}<{\centering}p{0.8cm}<{\centering}p{0.8cm}<{\centering}}
			
			\hline
			       \multirow{2}{*}{}                  & \multicolumn{3}{c}{$Accuracy^c$}\\
			                         \cline{2-4}
			                         & 2 & 3& Overall\\
			\hline
			\tool                    &  0.44 & 0.47  &   0.45        \\
			\tool w/o Context        &  0.38 & 0.43  &   0.40        \\
			\tool w/o Code Clone     &  0.42 & 0.44  &   0.43        \\
			\hline
		\end{tabular}
		\label{RQ4-result-1}
	\end{center}
\end{table}


Table~\ref{RQ4-result-1} shows the changes to the metrics as we remove one key feature from \tool. Generally, each key feature contributes positively to the better performance of {\tool}, as the decreasing of $Accuracy^c$ when removing each of them from \tool. When {\tool} removes context information, the $Accuracy^c$ decreased by $13.7\%, 8.5\%$, and $11.1\%$ on $2$ concerns data, $3$ concerns data, and overall performance, respectively. When {\tool} removes code clone, the $Accuracy^c$ decreased by $4.5\%, 6.4\%$, and $4.4\%$ on $2$ concerns data, $3$ concerns data, and overall performance, respectively.


\begin{table}[t]
	\caption{RQ4. Impact of the Number $k$ of Hops for Context}
	\vspace{-0.1in}
	\begin{center}
		\footnotesize
		\tabcolsep 4pt
		\renewcommand{\arraystretch}{1} \begin{tabular}{p{3cm}<{\centering}|p{0.8cm}<{\centering}p{0.8cm}<{\centering}p{0.8cm}<{\centering}}
			
			\hline
			       \multirow{2}{*}{}                  & \multicolumn{3}{c}{$Accuracy^c$}\\
\cline{2-4}
& 2 & 3& Overall\\
			\hline
			\tool ($k=1$)          & 0.42 & 0.45 &  0.43          \\
			\tool ($k=2$)          & 0.44 & 0.47 &  0.45          \\
			\tool ($k=3$)          & 0.43 & 0.45 &  0.44          \\
			\tool ($k=4$)          & 0.40 & 0.44 &  0.42          \\
			\tool ($k=5$)          & 0.41 & 0.44 &  0.42          \\
			\tool (Full Graph)     & 0.39 & 0.43 &  0.41          \\
			\hline
		\end{tabular}
		\label{RQ4-result-2}
	\end{center}
\end{table}

Table~\ref{RQ4-result-2} shows that selecting $k$-neighbors for context in \tool, $k=2$ is the best choice for \tool because of the highest $Accuracy^c$ on $2$ concerns data, $3$ concerns data and overall performance. Even though in this table, we only show the $k$ value range from $1$ to $5$. However, from the trend of $Accuracy^c$ changing, we can see that if $k>=2$, when $k$ increases, the $Accuracy^c$ decreases with the reason that when $k$ increases, the newly added nodes have less relationship with the entire node and may bring more biases. Based on this, even compared with the untested $k$ value, $k=2$ is still the best setting for \tool.
