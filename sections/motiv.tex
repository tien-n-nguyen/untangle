\section{Motivation}
\label{motiv:sec}

\subsection{Motivating Examples}
\label{exe:sec}



\begin{figure}[t]
	\centering
	\lstset{
		numbers=left,
		numberstyle= \tiny,
		keywordstyle= \color{blue!70},
		commentstyle= \color{red!50!green!50!blue!50},
		frame=shadowbox,
		rulesepcolor= \color{red!20!green!20!blue!20} ,
		xleftmargin=1.5em,xrightmargin=0em, aboveskip=1em,
		framexleftmargin=1.5em,
                numbersep= 5pt,
		language=Java,
    basicstyle=\scriptsize\ttfamily,
    numberstyle=\scriptsize\ttfamily,
    emphstyle=\bfseries,
                moredelim=**[is][\color{red}]{@}{@},
		escapeinside= {(*@}{@*)}
	}
	\begin{lstlisting}[]
private static IEnumerable<Tuple<SpecificationProperty, Maybe<Error>>> MapValuesImpl(...) {
  ...
  var pt = specProps.First();
(*@{\color{red}{   - var taken = values.Take(pt.Specification.CountOfMaxNumberOfValues().}@*) (*@{\color{red}{MapValueOrDefault(n => n, values.Count()));}@*)
(*@{\color{cyan}{   + var taken = values.Take(pt.Specification.CountOfMaxNumberOfValues().}@*) (*@{\color{cyan}{GetValueOrDefault(values.Count()));}@*)
   if (taken.Empty()) {...} ...
}
//--------------------------------------------------------------------------
public static TypeDescriptor WithNextValue(this TypeDescriptor descriptor, Maybe<TypeDescriptor> nextValue) {
(*@{\color{red}{  - return TypeDescriptor.Create(descriptor.TargetType,descriptor.MaxItems,}@*) (*@{\color{red}{nextValue.MapValueOrDefault(n => n, default(TypeDescriptor)));}@*)
(*@{\color{cyan}{  + return TypeDescriptor.Create(descriptor.TargetType,descriptor.MaxItems,}@*) (*@{\color{cyan}{nextValue.GetValueOrDefault(default(TypeDescriptor)));}@*)
}
//--------------------------------------------------------------------------
public static ParserResult<T> Build<T>(...) {
  ...
  select
(*@{\color{red}{  - sp.Value.MapValueOrDefault(}@*)
(*@{\color{red}{  -   v => v,}@*)
(*@{\color{red}{  -   sp.Specification.DefaultValue.MapValueOrDefault(}@*)
(*@{\color{red}{  -     d => d,}@*)
(*@{\color{cyan}{  + sp.Value.GetValueOrDefault(}@*)
(*@{\color{cyan}{  + \quad sp.Specification.DefaultValue.GetValueOrDefault(}@*)
(*@{\color{cyan}{   \quad  sp.Specification.ConversionType.CreateDefaultForImmutable()))).ToArray();}@*)
  var immutable = (T)ctor.Invoke(values);
  return immutable;
}
	\end{lstlisting}
        \vspace{-15pt}
        \caption{Co-Change}
        \vspace{-6pt}
        \label{fig:motiv-cc}
\end{figure}

\begin{figure}[t]
	\centering
	\lstset{
		numbers=left,
		numberstyle= \tiny,
		keywordstyle= \color{blue!70},
		commentstyle= \color{red!50!green!50!blue!50},
		frame=shadowbox,
		rulesepcolor= \color{red!20!green!20!blue!20} ,
		xleftmargin=1.5em,xrightmargin=0em, aboveskip=1em,
		framexleftmargin=1.5em,
                numbersep= 5pt,
		language=Java,
    basicstyle=\scriptsize\ttfamily,
    numberstyle=\scriptsize\ttfamily,
    emphstyle=\bfseries,
                moredelim=**[is][\color{red}]{@}{@},
		escapeinside= {(*@}{@*)}
	}
	\begin{lstlisting}[]
public override void Initialize() {
  ...
  // set algorithm framework models
(*@{\color{red}{- PortfolioSelection=new}@*) (*@{\color{red}{ManualPortfolioSelectionModel(QuantConnect.Symbol.Create("BTCUSD",}@*) (*@{\color{red}{SecurityType.Crypto, Market.GDAX));}@*)   
(*@{\color{cyan}{+ UniverseSelection=new}@*) (*@{\color{cyan}{ManualUniverseSelectionModel(QuantConnect.Symbol.Create("BTCUSD",}@*) (*@{\color{cyan}{SecurityType.Crypto, Market.GDAX));}@*)
  Alpha = new ConstantAlphaModel(InsightType.Price, InsightDirection.Up,...);
  PortfolioConstruction = new EqualWeightingPortfolioConstructionModel();
}
//--------------------------------------------------------------------------
public override void Initialize() {
  ...
  // set algorithm framework models
(*@{\color{red}{- PortfolioSelection=new}@*) (*@{\color{red}{ManualPortfolioSelectionModel(QuantConnect.Symbol.Create("SPY",}@*) (*@{\color{red}{SecurityType.Equity, Market.USA));}@*)   
(*@{\color{cyan}{+ UniverseSelection=new}@*) (*@{\color{cyan}{ManualUniverseSelectionModel(QuantConnect.Symbol.Create("SPY",}@*) (*@{\color{cyan}{SecurityType.Equity, Market.USA);}@*)
  Alpha = new ConstantAlphaModel(InsightType.Price, InsightDirection.Up,...);
  PortfolioConstruction = new EqualWeightingPortfolioConstructionModel();
//--------------------------------------------------------------------------
public override void Initialize() {
  ...
(*@{\color{red}{- PortfolioSelection = new CustomFundamentalPortfolioSelectionModel();}@*)
(*@{\color{cyan}{+ UniverseSelection = new CustomFundamentalUniverseSelectionModel();}@*)
  Alpha = new MacdAlphaModel(TimeSpan.FromMinutes(10),...);
  PortfolioConstruction = new EqualWeightingPortfolioConstructionModel();
}
	\end{lstlisting}
        \vspace{-15pt}
        \caption{Clone}
        \vspace{-6pt}
        \label{fig:motiv-clone}
\end{figure}

\begin{figure}[t]
	\centering
	\lstset{
		numbers=left,
		numberstyle= \tiny,
		keywordstyle= \color{blue!70},
		commentstyle= \color{red!50!green!50!blue!50},
		frame=shadowbox,
		rulesepcolor= \color{red!20!green!20!blue!20} ,
		xleftmargin=1.5em,xrightmargin=0em, aboveskip=1em,
		framexleftmargin=1.5em,
                numbersep= 5pt,
		language=Java,
    basicstyle=\scriptsize\ttfamily,
    numberstyle=\scriptsize\ttfamily,
    emphstyle=\bfseries,
                moredelim=**[is][\color{red}]{@}{@},
		escapeinside= {(*@}{@*)}
	}
	\begin{lstlisting}[]
public void mousePressed(MouseEvent evt) { 
  ...//SelectionTool.java
(*@{\color{red}{- if (figure != null) {}@*)
(*@{\color{cyan}{+ if (figure != null}@*) && (*@{\color{cyan}{figure.isSelectable()) {}@*)
       newTracker = createDragTracker(figure);
    } else {
       if (! evt.isShiftDown()) {...}
}
//--------------------------------------------------------------------------
protected void updateHoverHandles(DrawingView view, Figure f) {
  ...// SelectAreaTracker.java
  figure = f;
(*@{\color{red}{- if (figure != null) {}@*)
(*@{\color{cyan}{+ if (figure != null}@*) && (*@{\color{cyan}{figure.isSelectable()) {}@*)
      hoverHandles.addAll(figure.createHandles(-1));
    ...
}
	\end{lstlisting}
        \vspace{-15pt}
        \caption{Context}
        \vspace{-6pt}
        \label{fig:motiv-context}
\end{figure}

Let us present the real-world examples and our observations for
motivation.

