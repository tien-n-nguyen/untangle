\section{Motivation}
\label{motiv:sec}

\subsection{Motivating Examples}
\label{exe:sec}



\begin{figure}[t]
	\centering
	\lstset{
		numbers=left,
		numberstyle= \tiny,
		keywordstyle= \color{blue!70},
		commentstyle= \color{red!50!green!50!blue!50},
		frame=shadowbox,
		rulesepcolor= \color{red!20!green!20!blue!20} ,
		xleftmargin=1.5em,xrightmargin=0em, aboveskip=1em,
		framexleftmargin=1.5em,
                numbersep= 5pt,
		language=Java,
    basicstyle=\scriptsize\ttfamily,
    numberstyle=\scriptsize\ttfamily,
    emphstyle=\bfseries,
                moredelim=**[is][\color{red}]{@}{@},
		escapeinside= {(*@}{@*)}
	}
	\begin{lstlisting}[]
private static IEnumerable<Tuple<SpecificationProperty, Maybe<Error>>> MapValuesImpl(...) {
  ...
  var pt = specProps.First();
(*@{\color{red}{   - var taken = values.Take(pt.Specification.CountOfMaxNumberOfValues().}@*) (*@{\color{red}{MapValueOrDefault(n => n, values.Count()));}@*)
(*@{\color{cyan}{   + var taken = values.Take(pt.Specification.CountOfMaxNumberOfValues().}@*) (*@{\color{cyan}{GetValueOrDefault(values.Count()));}@*)
   if (taken.Empty()) {...} ...
}
//--------------------------------------------------------------------------
public static TypeDescriptor WithNextValue(this TypeDescriptor descriptor, Maybe<TypeDescriptor> nextValue) {
(*@{\color{red}{  - return TypeDescriptor.Create(descriptor.TargetType,descriptor.MaxItems,}@*) (*@{\color{red}{nextValue.MapValueOrDefault(n => n, default(TypeDescriptor)));}@*)
(*@{\color{cyan}{  + return TypeDescriptor.Create(descriptor.TargetType,descriptor.MaxItems,}@*) (*@{\color{cyan}{nextValue.GetValueOrDefault(default(TypeDescriptor)));}@*)
}
//--------------------------------------------------------------------------
public static ParserResult<T> Build<T>(...) {
  ...
  select
(*@{\color{red}{- sp.Value.MapValueOrDefault(v=>v,}@*) (*@{\color{red}{sp.Specification.DefaultValue.MapValueOrDefault(d => d,}@*)
(*@{\color{cyan}{  + sp.Value.GetValueOrDefault(}@*)
(*@{\color{cyan}{  + \quad sp.Specification.DefaultValue.GetValueOrDefault(}@*)
(*@{\color{cyan}{   \quad  sp.Specification.ConversionType.CreateDefaultForImmutable()))).ToArray();}@*)
  var immutable = (T)ctor.Invoke(values);
  return immutable;
}
	\end{lstlisting}
        \vspace{-15pt}
        \caption{Co-Changed Statements for the Same Concern}
        \vspace{-6pt}
        \label{fig:motiv-cc}
\end{figure}

%public static ParserResult<T> Build<T>(...) {
%  ...
%  select
%(*@{\color{red}{  - sp.Value.MapValueOrDefault(}@*)
%(*@{\color{red}{  - \quad   v => v,}@*)
%(*@{\color{red}{  - \quad  sp.Specification.DefaultValue.MapValueOrDefault(}@*)
%(*@{\color{red}{  - \quad \quad   d => d,}@*)
%(*@{\color{cyan}{  + sp.Value.GetValueOrDefault(}@*)
%(*@{\color{cyan}{  + \quad sp.Specification.DefaultValue.GetValueOrDefault(}@*)
%(*@{\color{cyan}{   \quad  sp.Specification.ConversionType.CreateDefaultForImmutable()))).ToArray();}@*)
%  var immutable = (T)ctor.Invoke(values);
%  return immutable;
%}

Let us present the real-world examples and our observations for
motivation. Figure~\ref{fig:motiv-cc} shows an example from the
experimental dataset in a prior work~\cite{flexeme-fse20}. This commit
has three changed statements at lines 4, 10, and 17 of the methods
\code{MapValuesImpl}, \code{WithNextValue}, and \code{Build}.  The
method \code{MapValueOrDefault} was renamed to
\code{GetValueOr\-Default}, causing the changes to the call sites.
These changes are deemed by the developers as serving to the same
concern/purpose of enhancing type resolution in the parser.  These
statements have been changed for the same concern in a past
commit. This motivates us to build a machine learning (ML) model to
learn from the co-changes for the same concern in the version history
to untangle the current commit. For this example, the untangling
approaches relying on the PDGs or program slices within individual
methods cannot cluster the changed statements into a group.

%(e.g.,~\cite{flexeme-fse20},~\cite{roover-scam18})

\noindent {\bf Observation 1 [Learn to Cluster Code Changes].} {\em
  History of the co-changed statements for the same concern could be a
  good source for an ML model to learn to cluster the changed statements, thus,
  untangling the current commits}.

\begin{figure}[t]
	\centering
	\lstset{
		numbers=left,
		numberstyle= \tiny,
		keywordstyle= \color{blue!70},
		commentstyle= \color{red!50!green!50!blue!50},
		frame=shadowbox,
		rulesepcolor= \color{red!20!green!20!blue!20} ,
		xleftmargin=1.5em,xrightmargin=0em, aboveskip=1em,
		framexleftmargin=1.5em,
                numbersep= 5pt,
		language=Java,
    basicstyle=\scriptsize\ttfamily,
    numberstyle=\scriptsize\ttfamily,
    emphstyle=\bfseries,
                moredelim=**[is][\color{red}]{@}{@},
		escapeinside= {(*@}{@*)}
	}
	\begin{lstlisting}[]
public void mousePressed(MouseEvent evt) { 
  ...//SelectionTool.java
(*@{\color{red}{- if (figure != null) {}@*)
(*@{\color{cyan}{+ if (figure != null}@*) && (*@{\color{cyan}{figure.isSelectable()) {}@*)
       newTracker = createDragTracker(figure);
    } else {
       if (! evt.isShiftDown()) {...}
}
//--------------------------------------------------------------------------
protected void updateHoverHandles(DrawingView view, Figure f) {
  ...// SelectAreaTracker.java
  figure = f;
(*@{\color{red}{- if (figure != null) {}@*)
(*@{\color{cyan}{+ if (figure != null}@*) && (*@{\color{cyan}{figure.isSelectable()) {}@*)
      hoverHandles.addAll(figure.createHandles(-1));
    ...
}
	\end{lstlisting}
        \vspace{-15pt}
        \caption{Same Change in Two Contexts for Different Concerns}
        \vspace{-6pt}
        \label{fig:motiv-context}
\end{figure}

Figure~\ref{fig:motiv-context} shows another example in the project
JHotDraw. At the commit rev. 463, two changes at line 3 and line 13
are exactly the same with the addition of the checking whether a
figure is selectable via \code{figure.isSelectable()}. However, those
{\em two exact changes} occurred at two different methods
\code{mousePressed} and \code{updateHoverHandles} for two different
concerns/purposes as noted in the commit log. Line 4 was aimed to fix
the tracking of a dragging object as the mouse is pressed ({\em
  ``Fixed bug where setSelectable() on a dragging figure did not
  work''}).  Line 14 was a fix for a different bug as the hovered
object needs to be selected ({\em ``Selection classes now check the
  flag on the hovering figures to be selected.''}). The addition of
\code{figure.isSelectable()} is common for checking if a figure is
selectable in the two tasks with dragging figure and hovering
figure. Without the surrounding contexts, one could not tell they are
for different purposes.

\noindent {\bf Observation 2 [Context].} {\em The surrounding context
  is important to determine the purpose/concern of a change. The same
  change in two different contexts might be for different
  purposes/concerns}.

\begin{figure}[t]
	\centering
	\lstset{
		numbers=left,
		numberstyle= \tiny,
		keywordstyle= \color{blue!70},
		commentstyle= \color{red!50!green!50!blue!50},
		frame=shadowbox,
		rulesepcolor= \color{red!20!green!20!blue!20} ,
		xleftmargin=1.5em,xrightmargin=0em, aboveskip=1em,
		framexleftmargin=1.5em,
                numbersep= 5pt,
		language=Java,
    basicstyle=\scriptsize\ttfamily,
    numberstyle=\scriptsize\ttfamily,
    emphstyle=\bfseries,
                moredelim=**[is][\color{red}]{@}{@},
		escapeinside= {(*@}{@*)}
	}
	\begin{lstlisting}[]
public override void Initialize() {
  ...
  // set algorithm framework models
(*@{\color{red}{- PortfolioSelection=new}@*) (*@{\color{red}{ManualPortfolioSelectionModel(QuantConnect.Symbol.Create("BTCUSD",}@*) (*@{\color{red}{SecurityType.Crypto, Market.GDAX));}@*)   
(*@{\color{cyan}{+ UniverseSelection=new}@*) (*@{\color{cyan}{ManualUniverseSelectionModel(QuantConnect.Symbol.Create("BTCUSD",}@*) (*@{\color{cyan}{SecurityType.Crypto, Market.GDAX));}@*)
  Alpha = new ConstantAlphaModel(InsightType.Price, InsightDirection.Up,...);
  PortfolioConstruction = new EqualWeightingPortfolioConstructionModel();
}
//--------------------------------------------------------------------------
public override void Initialize() {
  ...
  // set algorithm framework models
(*@{\color{red}{- PortfolioSelection=new}@*) (*@{\color{red}{ManualPortfolioSelectionModel(QuantConnect.Symbol.Create("SPY",}@*) (*@{\color{red}{SecurityType.Equity, Market.USA));}@*)   
(*@{\color{cyan}{+ UniverseSelection=new}@*) (*@{\color{cyan}{ManualUniverseSelectionModel(QuantConnect.Symbol.Create("SPY",}@*) (*@{\color{cyan}{SecurityType.Equity, Market.USA);}@*)
  Alpha = new ConstantAlphaModel(InsightType.Price, InsightDirection.Up,...);
  PortfolioConstruction = new EqualWeightingPortfolioConstructionModel();
	\end{lstlisting}
        \vspace{-15pt}
        \caption{Cloned Code in the Same Concern}
        \vspace{-3pt}
        \label{fig:motiv-clone}
\end{figure}

%//--------------------------------------------------------------------------
%public override void Initialize() {
%  ...
%(*@{\color{red}{- PortfolioSelection = new CustomFundamentalPortfolioSelectionModel();}@*)
%(*@{\color{cyan}{+ UniverseSelection = new CustomFundamentalUniverseSelectionModel();}@*)
%  Alpha = new MacdAlphaModel(TimeSpan.FromMinutes(10),...);
%  PortfolioConstruction = new EqualWeightingPortfolioConstructionModel();
%}

Figure~\ref{fig:motiv-clone} shows another example on the two
fragments of cloned code for two different security types (Crypto and
Equity) and markets (GDAX and USA). They were fixed in the same manner
at line 4 and line 13. The two changes were in the same fix to convert
the usage of \code{PortfolioSelection} into that of
\code{UniverseSelection}, and the usage of
\code{ManualPortfolioSelectionModel} into that of
\code{ManualUniverseSelection\-Model} in the process of {\em ``setting
  algorithm framework models''}. The two changes belong to the same
concern/purpose since they serve as a fix for the same logic despite
that they have no program dependencies among one another. The existing
untangling
approaches~\cite{flexeme-fse20,smartcommit-fse21,roover-scam18,barnett-icse15}
that require the explicit program dependencies among the changes with
the same purpose will not classify these committed code as belonging
to the same concern.

\noindent {\bf Observation 3 [Implicit Dependencies].} {\em Two
  fragments of cloned code have the same/similar logic, thus, have
  implicit dependencies, and could be modified in the same manner to
  serve for the same purpose/concern.}

%\vspace{-12pt}
\subsection{Key Ideas}
\label{ideas:sec}

From the motivation, we propose {\tool} with the following
ideas.

%{\bf Key Idea 1 [Code Change Representation Learning with Graph
%    Convolution Network (GCN)]}.

{\bf Key Idea 1 [Code Change Clustering Learning Model]}. Instead of
deciding a deterministic clustering criterion on the concrete
artifacts (PDGs, program slices, code changes, operations, or change
graphs), from Observation 1, we build an ML model to untangle the
commits by learning to cluster code~changes represented by embeddings,
w.r.t. different concerns. The model learns from the~history of the
co-changed statements in the same commits for the same concerns, and
applies to cluster the changes in the current commit.  We also modify
an agglomerative clustering algorithm into a~super\-vised-learning
clustering model via trainable parameters and a loss function to
compare the predicted clusters and the correct ones.

{\bf Key Idea 2 [Context-aware, Graph-based Representation Learning
    for Code Changes].} We design a context-aware, graph-based,
representation learning model to {\em learn the contextualized
  embeddings (vectors) for the code changes} that integrates {\em
  program dependencies} among the program elements, and the {\em
  contexts} of~code changes. We train a Label, Graph-based Convolution
Network~\cite{label-gcn} to learn the embeddings and learn the
clusters from the history of changed code for the same concerns.
For prediction, we apply supervised-learning agglomerative
clustering on the embeddings of code changes to produce the clusters
to untangle the commit. We expect that supervised-learning clustering
on vectors is more effective than on the above artifacts since the
embeddings capture~richer information integrated from the
dependencies and contexts.


{\bf Key Idea 3 [Explicit Context Representation as a Weight to
    Compute Vectors for Code Changes]}. As in Observation 2, the
contexts of the code changes can help distinguish their concerns in
the commits. We represent code changes and the surrounding context of
a change via the multi-version program dependence~graph,
$\delta$-PDG~\cite{flexeme-fse20}, consisting of the elements of both
versions before and after the changes, and program dependencies. The
context is defined as the surrounding nodes of the changed statement
node in that graph. The Label-GCN is used to model the statements and their
dependencies in $\delta$-PDG and to learn the vector
representation of the context for a change. The context vectors are
then used as weights in learning contextualized embeddings for the
code changes.

{\bf Key Idea 4 [Implicit Dependencies among Cloned Code]}. As in
Observation 3, the cloned code exhibits implicit dependencies with
regard to whether they can be changed in the same commits for the same
concerns. Thus, during the process of producing the final
result, we also integrate the code clone relationships to
adjust the clusters produced by the clustering learning model.

%Thus, the GCN model can learn the embeddings of the code changes with
%the consideration of the cloned code.



