\section{Motivation}
\label{motiv:sec}

\subsection{Motivating Examples}
\label{exe:sec}

\begin{figure}[t]
	\centering
	\lstset{
		numbers=left,
		numberstyle= \tiny,
		keywordstyle= \color{blue!70},
		commentstyle= \color{red!50!green!50!blue!50},
		frame=shadowbox,
		rulesepcolor= \color{red!20!green!20!blue!20},
                rulecolor= \color{black},
		xleftmargin=1.5em,xrightmargin=0em, aboveskip=1em,
		framexleftmargin=1.5em,
                numbersep= 5pt,
		language=Java,
    basicstyle=\scriptsize\ttfamily,
    numberstyle=\scriptsize\ttfamily,
    emphstyle=\bfseries,
                moredelim=**[is][\color{red}]{@}{@},
		escapeinside= {(*@}{@*)}
	}
	\begin{lstlisting}[]
Commit r1192 ``Fixes NPE and implements indentation of XML elements.''
 /trunk/jhotdraw8/src/main/java/org/jhotdraw8/draw/Drawing.java
 ...
(*@{\color{red}{- public final static Key<List<URI>> AUTHOR\_STYLESHEETS = new SimpleFigureKey<> ("authorStylesheets", List.class, new Class<?>[]\{URI.class\}},...,}@*)(*@{\color{violet}{ null);}@*)
(*@{\color{cyan}{+ public final static Key<List<URI>> AUTHOR\_STYLESHEETS = new SimpleFigureKey<> ("authorStylesheets", List.class, new Class<?>[]\{URI.class\}},...,}@*)(*@{\color{violet}{ Collections.emptyList());}@*)
 ...
(*@{\color{red}{- public final static Key<List<URI>> USER\_AGENT\_STYLESHEETS = new SimpleFigureKey<>("userAgentStylesheets", List.class, new Class<?>[]{URI.class},...,}@*)(*@{\color{violet}{ null);}@*)
(*@{\color{cyan}{+ public final static Key<List<URI>> USER\_AGENT\_STYLESHEETS = new SimpleFigureKey<> ("userAgentStylesheets", List.class, new Class<?>[]\{URI.class\},...,}@*)(*@{\color{violet}{ Collections.emptyList());}@*)
 ...
(*@{\color{red}{- public final static Key<List<String>> INLINE\_STYLESHEETS=new SimpleFigureKey <>("inlineStylesheets",List.class,new Class<?>[]\{String.class\},.,}@*)(*@{\color{violet}{null);}@*)
(*@{\color{cyan}{+ public final static Key<List<String>> INLINE\_STYLESHEETS=new SimpleFigureKey <>("inlineStylesheets", List.class, new Class<?>[]\{String.class\},...,}@*)(*@{\color{violet}{ Collections.emptyList());}@*)
//--------------------------------------------------------------------------
 /trunk/jhotdraw8/src/main/java/org/jhotdraw8/draw/io/SimpleXmlIO.java
 public Document toDocument(Drawing internal,Collection<Figure> selection)..{
   for (Figure child : ordered) {
   (*@{\color{red}{-            writeNodeRecursively(doc, docElement, child);}@*)
   (*@{\color{red}{-\}}@*)
   (*@{\color{red}{-          docElement.appendChild(doc.createTextNode("..."));}@*)
   (*@{\color{cyan}{+            writeNodeRecursively(doc, docElement, child, linebreak);}@*)
   (*@{\color{cyan}{+\}}@*)
   (*@{\color{cyan}{+          docElement.appendChild(doc.createTextNode(linebreak));}@*)
   return doc; ...
}
	\end{lstlisting}
        \vspace{-15pt}
        \caption{A Tangled Commit at r1192 of JHotDraw}
%        \vspace{-6pt}
        \label{fig:motiv-cc}
\end{figure}

\begin{figure}[t]
	\centering
	\lstset{
		numbers=left,
		numberstyle= \tiny,
		keywordstyle= \color{blue!70},
		commentstyle= \color{red!50!green!50!blue!50},
		frame=shadowbox,
		rulesepcolor= \color{red!20!green!20!blue!20} ,
		xleftmargin=1.5em,xrightmargin=0em, aboveskip=1em,
		framexleftmargin=1.5em,
                numbersep= 5pt,
		language=Java,
    basicstyle=\scriptsize\ttfamily,
    numberstyle=\scriptsize\ttfamily,
    emphstyle=\bfseries,
                moredelim=**[is][\color{red}]{@}{@},
		escapeinside= {(*@}{@*)}
	}
	\begin{lstlisting}[]
  Commit r1023 ``Fixes bugs in FigureStyleManager''
  /trunk/jhotdraw8/src/main/java/org/jhotdraw8/draw/Drawing.java
   
(*@{\color{red}{-    public final static Key<List<URI>> AUTHOR\_STYLESHEETS = new SimpleFigureKey<> ("authorStylesheets", List.class, }@*)(*@{\color{violet}{"<URI>",}@*)(*@{\color{red}{,..., null);}@*)
(*@{\color{cyan}{+     public final static Key<List<URI>> AUTHOR\_STYLESHEETS = new SimpleFigureKey<> ("authorStylesheets", List.class, }@*)(*@{\color{violet}{new Class<?>[]\{URI.class\},}@*)(*@{\color{cyan}{..., null);}@*)

(*@{\color{red}{-    public final static Key<List<URI>> USER\_AGENT\_STYLESHEETS=new SimpleFigureKey <>("userAgentStylesheets", List.class, }@*)(*@{\color{violet}{"<URI>",}@*)(*@{\color{red}{..., null);}@*)
(*@{\color{cyan}{+ public final static Key<List<URI>> USER\_AGENT\_STYLESHEETS=new SimpleFigureKey<> ("userAgentStylesheets", List.class, }@*)(*@{\color{violet}{new Class<?>[]\{URI.class\},}@*)(*@{\color{cyan}{..., null);}@*)

(*@{\color{red}{-    public final static Key<List<String>> INLINE\_STYLESHEETS=new SimpleFigureKey <>("inlineStylesheets", List.class, }@*)(*@{\color{violet}{"<String>",}@*)(*@{\color{red}{..., null);}@*)
(*@{\color{cyan}{+    public final static Key<List<String>> INLINE\_STYLESHEETS=new SimpleFigureKey <>("inlineStylesheets",List.class,}@*)(*@{\color{violet}{new Class<?>[]\{String.class\},}@*)(*@{\color{cyan}{.,null);}@*)
	\end{lstlisting}
        \vspace{-15pt}
        \caption{Same Statements as in r1192 were Changed at r1023 of JHotDraw and Belonged to Only One Concern}
        \vspace{-6pt}
        \label{fig:history}
\end{figure}

%\caption{Changes of Same Concern in r1023 of JHotDraw}

%public static ParserResult<T> Build<T>(...) {
%  ...
%  select
%(*@{\color{red}{  - sp.Value.MapValueOrDefault(}@*)
%(*@{\color{red}{  - \quad   v => v,}@*)
%(*@{\color{red}{  - \quad  sp.Specification.DefaultValue.MapValueOrDefault(}@*)
%(*@{\color{red}{  - \quad \quad   d => d,}@*)
%(*@{\color{cyan}{  + sp.Value.GetValueOrDefault(}@*)
%(*@{\color{cyan}{  + \quad sp.Specification.DefaultValue.GetValueOrDefault(}@*)
%(*@{\color{cyan}{   \quad  sp.Specification.ConversionType.CreateDefaultForImmutable()))).ToArray();}@*)
%  var immutable = (T)ctor.Invoke(values);
%  return immutable;
%}

Let us present a few real-world examples to motivate {\tool}.
%our motivation.

\vspace{-6pt}
\subsubsection{Example 1}
\label{sec:example-1}

Figure~\ref{fig:motiv-cc} shows the changes at the commit r1192 of the
JHotDraw project with the log {\em ``Fixes NPE and implements
  indentation of~XML elements''}. This is a tangled commit with two
concerns/purposes: 1) the fix for Null Pointer Exception occurred at
lines 4,7, and 10 of the \code{Drawing} class (the \code{null}
argument was replaced with \code{Collections.empty\-List()}); 2) the
implementation of the indentation of XML elements occurred at lines
16--18, and a few other lines of code in the \code{SimpleXmlIO} class
and one line of code in the \code{Drawing} class (not shown).

\subsubsection{Example 2}

Figure~\ref{fig:history} shows the changes committed at r1023 earlier
in JHotDraw. The changes at the lines 4, 7, and 10 of the
\code{Drawing} class were to the same statements as the ones in the
r1192 commit in Figure~\ref{fig:motiv-cc}. However, the commit at
r1023 was for only one concern as stated in the commit log {\em
  ``Fixes bugs in FigureStyleManager''}. This example motivates us to
build a ML model to learn from the co-changes for the same concern in
the version history to untangle the current commit.

%Figure~\ref{fig:motiv-cc} shows an example from the experimental
%dataset in a prior work~\cite{flexeme-fse20}. This commit has three
%changed statements at lines 4, 10, and 17 of the methods
%\code{MapValuesImpl}, \code{WithNextValue}, and \code{Build}.  The
%method \code{MapValueOrDefault} was renamed to
%\code{GetValueOr\-Default}, causing the changes to the call sites.
%These changes are deemed by the developers as serving the same
%concern/purpose of enhancing type resolution in the parser. {\em These
%  statements have been changed for the same concern} of fixing the
%value parsing in a past commit. This motivates us to build a ML model
%to learn from the co-changes for the same concern in the version
%history to untangle the commits.
%---------------------

%Tien
%For this example, the untangling approaches relying on the PDGs or
%program slices within individual methods cannot cluster the changed
%statements into a group.

%(e.g.,~\cite{flexeme-fse20},~\cite{roover-scam18})

\vspace{3pt}
\noindent {\bf Observation 1 [Learn to Cluster Code Changes].} {\em
  History of the co-changed statements with the same concern could be
  a good source for a machine learning model to learn to cluster the
  changed statements, thus, untangling the current commit}.

\begin{figure}[t]
	\centering
	\lstset{
		numbers=left,
		numberstyle= \tiny,
		keywordstyle= \color{blue!70},
		commentstyle= \color{red!50!green!50!blue!50},
		frame=shadowbox,
		rulesepcolor= \color{red!20!green!20!blue!20} ,
		xleftmargin=1.5em,xrightmargin=0em, aboveskip=1em,
		framexleftmargin=1.5em,
                numbersep= 5pt,
		language=Java,
    basicstyle=\scriptsize\ttfamily,
    numberstyle=\scriptsize\ttfamily,
    emphstyle=\bfseries,
                moredelim=**[is][\color{red}]{@}{@},
		escapeinside= {(*@}{@*)}
	}
	\begin{lstlisting}[]
public void mousePressed(MouseEvent evt) { 
  ...//SelectionTool.java
(*@{\color{red}{- if (figure != null) {}@*)
(*@{\color{cyan}{+ if (figure != null}@*) && (*@{\color{cyan}{figure.isSelectable()) {}@*)
       newTracker = createDragTracker(figure);
    } else {
       if (! evt.isShiftDown()) {...}
}
//--------------------------------------------------------------------------
protected void updateHoverHandles(DrawingView view, Figure f) {
  ...// SelectAreaTracker.java
  figure = f;
(*@{\color{red}{- if (figure != null) {}@*)
(*@{\color{cyan}{+ if (figure != null}@*) && (*@{\color{cyan}{figure.isSelectable()) {}@*)
      hoverHandles.addAll(figure.createHandles(-1)); ...
}
	\end{lstlisting}
        \vspace{-15pt}
        \caption{Same Change in two Contexts for Different Concerns at r463 of JHotDraw}
%        \vspace{-6pt}
        \label{fig:motiv-context}
\end{figure}

% \caption{Same Change in Two Contexts for Different Concerns in the Commit r463 of JHotDraw}

\subsubsection{Example 3}

Figure~\ref{fig:motiv-context} shows another example in JHotDraw
project. At the commit r463, two changes at line 3 and line 13 are
exactly the same with the addition of \code{figure.isSelectable()} to
check whether a figure is selectable or not. However, those {\em two
  exact changes} occurred in two different methods \code{mousePressed}
and \code{updateHoverHandles} for two different concerns/purposes as
noted in the commit log. Line 4 was aimed to fix the tracking of a
dragging object as the mouse is pressed ({\em ``Fixed bug where
  setSelectable() on a dragging figure did not work''}).  Line 14 was
a fix for a different bug as the hovered object needs to be selected
({\em ``Selection classes now check the flag on the hovering figures
  to be selected.''}). The addition of \code{figure.isSelectable()} is
common for checking if a figure is selectable in the two tasks of
dragging and hovering a figure. Without the surrounding contexts,
one could not tell whether two changes are for the same purpose or not.

\vspace{3pt}
\noindent {\bf Observation 2 [Context].} {\em The surrounding context
  consisting of un-changed code is crucial to determine the
  concern of a change. The same change in two 
  contexts could be for different concerns}.

\vspace{3pt}
Figure~\ref{fig:motiv-cc} also gives us an interesting
observation. The changes to fix the NPE at lines 4, 7, and 10 are
exactly the same: \code{null} $\rightarrow$
\code{Collections.emptyList()}.  The three statements are cloned code
of one another to support for different stylesheets (author, user
agent, and inline stylesheets).  In fact, the changes in
Figure~\ref{fig:history} also occurred at the cloned statements. The
three changes belong to the same concern/purpose since they serve as a
fix for the same logic despite that there is no program dependency
among them. The existing untangling
approaches~\cite{flexeme-fse20,smartcommit-fse21,roover-scam18,barnett-icse15}
that require the explicit program dependencies among the changes with
the same purpose will not classify these committed code as belonging
to the same concern.

%Figure~\ref{fig:motiv-clone} shows another example on the two
%fragments of cloned code for two different security types (Crypto and
%Equity) and markets (GDAX and USA). They were fixed in the same manner
%at line 4 and line 13. The two changes were in the same fix to convert
%the usage of \code{PortfolioSelection} into that of
%\code{UniverseSelection}, and the usage of
%\code{ManualPortfolioSelectionModel} into that of
%\code{ManualUniverseSelection\-Model} in the process of {\em ``setting
%  algorithm framework models''}.

\vspace{2pt}
\noindent {\bf Observation 3 [Implicit Dependencies].} {\em Two
  fragments of cloned code have the same/similar logic, thus, have
  implicit dependencies, and could be modified in the same manner to
  serve the same purpose.}

%\vspace{-12pt}
\subsection{Key Ideas}
\label{ideas:sec}

From the motivation, we propose {\tool} with the following
ideas.

%{\bf Key Idea 1 [Code Change Representation Learning with Graph
%    Convolution Network (GCN)]}.

{\bf Key Idea 1 [Code Change Clustering Learning Model]}. Instead of
deciding a deterministic clustering criterion on the concrete
artifacts (PDGs, program slices, code changes, operations, or change
graphs), from Observation 1, we build an ML model to untangle the
commits by learning to cluster code~changes represented by embeddings,
w.r.t. different concerns. The model learns from the~history of the
co-changed statements in the same commits for the same concerns, and
applies to cluster the changes in the current commit.  We also modify
an agglomerative clustering algorithm into a~super\-vised-learning
clustering model via trainable parameters and a loss function to
compare the predicted clusters and the correct ones.

{\bf Key Idea 2 [Context-aware, Graph-based Representation Learning
    for Code Changes].} We design a context-aware, graph-based,
representation learning model to {\em learn the contextualized
  embeddings (vectors) for the code changes} that integrates {\em
  program dependencies} among the program elements, and the {\em
  contexts} of~code changes. We train a Label, Graph-based Convolution
Network~\cite{label-gcn} to learn the embeddings and learn the
clusters from the history of changed code for the same concerns.
For prediction, we apply supervised-learning agglomerative
clustering on the embeddings of code changes to produce the clusters
to untangle the commit. We expect that supervised-learning clustering
on vectors is more effective than on the above artifacts since the
embeddings capture~richer information integrated from the
dependencies and contexts.


{\bf Key Idea 3 [Explicit Context Representation as a Weight to
    Compute Vectors for Code Changes]}. As in Observation 2, the
contexts of the code changes can help distinguish their concerns in
the commits. We represent code changes and the surrounding context of
a change via the multi-version program dependence~graph,
$\delta$-PDG~\cite{flexeme-fse20}, consisting of the elements of both
versions before and after the changes, and program dependencies. The
context is defined as the surrounding nodes of the changed statement
node in that graph. The Label-GCN is used to model the statements and their
dependencies in $\delta$-PDG and to learn the vector
representation of the context for a change. The context vectors are
then used as weights in learning contextualized embeddings for the
code changes.

{\bf Key Idea 4 [Implicit Dependencies among Cloned Code]}. As in
Observation 3, the cloned code exhibits implicit dependencies with
regard to whether they can be changed in the same commits for the same
concerns. Thus, during the process of producing the final
result, we also integrate the code clone relationships to
adjust the clusters produced by the clustering learning model.

%Thus, the GCN model can learn the embeddings of the code changes with
%the consideration of the cloned code.



