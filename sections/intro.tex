\section{Introduction}
\label{intro:sec}

During software evolution, developers make changes over time
%and commit them into a source code repository.
to perform software maintenance tasks.
The changes to source files committed to the repository in the same transaction are referred to as a {\em change set} or a {\em commit}.
For separation of concerns, each commit should be about one
purpose~or~concern regarding the programming task at hand.
Unfortunately, it has been reported that many commits tangle different  concerns including the changes for bug-fixing, refactoring,
enhancements, improvements, or
documentation~\cite{tao-fse12,kim-emse16,kim-msr13,hill-tse12,nguyen-issre13}.
%Tao {\em et al.}~\cite{tao-fse12}, Kim {\em et
%  al.}~\cite{kim-emse16,kim-msr13}, and Hill {\em et
%  al.}~\cite{hill-tse12} have reported that many change sets tangle
%different concerns including the changes for bug-fixing, refactoring,
%enhancements/improvements, or documentation.
Such change sets are called {\em tangled code changes} or {\em tangled
  commits}~\cite{kim-emse16,kim-msr13}. The prior work reported two
reasons for tangled commits from developers' perspective: time
pressure in committing the changes, and unclear relations between the
concerns for code changes~\cite{flexeme-fse20}.

Tangled commits pose several issues in software development. First,
they affect software quality as they hamper program
comprehension~\cite{tao-fse12} and reduce the separation of concerns
in code changes \cite{flexeme-fse20}. Second, the tangled commits
might contain the bug-fixing changes for one bug that are mixed with
the fixes for other bugs as well as other types of changes for
refactoring, enhancements, or
documentation~\cite{kim-emse16,kim-msr13,nguyen-issre13}. Those
tangled commits have negative impact on the accuracy of bug prediction
or bug localization models that rely on the changes mined from the
repository~\cite{kim-emse16,kim-msr13}. Those models are
significantly affected by the tangled commits as they consider an
entire commit as for fixing or non-fixing.


%Those models consider an entire commit as for fixing or non-fixing, thus, are significantly affected by the tangled commits.

Recognizing the need of tools that untangle, i.e., decompose a commit
into untangled changes, researchers have proposed several
approaches. The automated commit-untangling approaches can be broadly
classified into two categories: {\em mining software
  repositories}~\cite{kim-msr13,kim-emse16,higo-apsec16,
  higo-icpc14,dias-saner15}, and {\em program
  analysis}~\cite{roover-scam18,barnett-icse15,flexeme-fse20,smartcommit-fse21}.

First, the earlier approaches leveraged {\em mining software repositories
  (MSR)} techniques to untangle commits. Herzig {\em et
  al.}~\cite{kim-msr13,kim-emse16} utilize a confidence voter
technique with agglomerative clustering on change
operations for untangling.
%the commits.
%Each confidence voter is responsible for an important aspect
%including call-graphs, change couplings, data dependencies, and
%distance measures.
%However, the voters are independent, thus, do not reflect well the
%interdependency~nature of program elements under change.
The voters consider data dependencies, call graphs, change
couplings, and distances.
%
Kirinuki {\em et al.}~\cite{higo-apsec16, higo-icpc14} consider a
commit as tangled if it includes another commit in the past. However,
there are tangled commits whose parts have not occurred
in the past.
%rely on the histories of the co-changes to split the tangled code
%changes before they are committed. However, they do not consider the
%relations among the changes such as data or control dependencies.
%
Dias {\em et al.}~\cite{dias-saner15} use confidence voters on the
fine-grained change events in an editor. The confidence scores are
converted into the similarity scores via a Random Forest Regressor,
which are then used in an agglomerative clustering algorithm to partition
the tangled changes.

The second category of untangling approaches leverage the {\em static
analysis} techniques. Roover {\em et
  al.}~\cite{roover-scam18} use program slicing to segment a commit
across a Program Dependency Graph (PDG).  However, it is limited in
handling inter-procedural and cross-file dependencies. Barnett {\em et
  al.}~\cite{barnett-icse15} use def-use chains, and cluster them. If
the def-use chain all falls into a method, it is considered as
trivial, otherwise, non-trivial. Because ignoring the trivial
clusters, it can miss tangled concerns. Flexeme~\cite{flexeme-fse20}
uses multi-version PDG augmented with name flows in the edges, and
applies agglomerative clustering using graph similarity on that graph
to untangle the commits. SmartCommit~\cite{smartcommit-fse21} uses a
graph-partitioning algorithm on a graph representation to capture the
relations among code changes (hard and soft links, refactoring links,
cosmetic links, etc.).

%the targets of their clustering algorithms are either changes, change
%operations, change events, or slices, PDGs, which do not sufficiently
%contain the information for untangling commits. The reason is that
%the boundaries across the concerns in a commit do not necessarily
%and natually map to a specific clustering criteria on those targets
%such as changes or PDGs. First, the boundaries across the concerns
%in a commit do not neccessarily and natually map to a clustering
%criteria on the PDG (with/without name flows).

%The concerns might be linked with multiple edges, and a statement
%might belong to multiple concerns.

%These points make the clustering algorithms difficult to specify the
%clustering criteria to achieve the best partitioning.

%\vspace{3pt}
Despite their successes, the state-of-the-art untangling commit
techniques still have limitations. First, the boundaries across concerns 
in a commit do not necessarily and naturally map to
clustering criteria of a clustering algorithm running on the PDG, name
flows, program slices, change operations, or the changes
themselves. The concerns might be linked via multiple edges.
%, and a statement might belong to multiple concerns.
To apply a clustering algorithm on the PDG, program slices, or change
graphs, it is challenging to deterministically specify the right
criteria to obtain perfect partitions matching with the concerns.
%Applying a clustering algorithm on the PDG, slices, or change graphs
%makes it difficult to specify clustering criteria to achieve the
%partitions matching with the concerns.
%
Second, the goal is to decompose the changes in a commit. However, the
existing approaches {\em do~not~consider a change w.r.t. the context
  of surrounding code with a clear distinction of the changed elements
  and the un-changed ones in the context}. Such context could help
distinguish the concerns for the changes. Finally, not all the changes
in the same concern need to have program dependencies among them. The
logic connection among the co-changed code in the same commit could be
due to the reasons different than program dependencies. For
example,~two pieces of cloned code realizing the same bubble sorting
algorithm have the same bug, e.g., at the comparison operator. They
might be changed for the same concern to fix that logic bug despite
that they have no data/control dependency.


%We develop {\tool}, a {\em novel code change clustering learning
%model} that learns to cluster the code changes, represented by the
%embeddings, into different groups with different concerns. The core of
%{\tool} is {\em context-aware, graph-based, code change representation
%learning (RL) model} leveraging Graph-based Convolution Network to
%produce the {\em contextualized embeddings (vectors)} for the code
%changes, that integrates program dependencies, the surrounding
%contexts of the changes, as well as the code clone relations. The
%contexts of the changes and cloned code are explicitly represented,
%helping the model distinguish their concerns/purposes.

%a machine learning-based approach that learns to untangle the changes
%in a commit into different clusters for different concerns.

%{\tool} is designed with the key ideas to address the above
%challenges.

%{\tool} represents the changes and the surrounding code via the
%multi-version program dependency graph, $\delta$-PDG (adapted from
%Flexeme~\cite{flexeme-fse20}) and uses the Graph Convolutional Network
%(GCN) to model the statements and their interdependencies in
%$\delta$-PDG.

%Finally, we design a novel \underline{code change representation learning}
%that integrates the graph structures in GCN for program dependencies,
%the representation of a change and its context, and the cloned code
%into the vector representations (i.e., embeddings) for the code
%changes in a commit.

%Tien

%\vspace{3pt}
To address those challenges, in this paper, we propose {\bf \tool}, a
{\bf novel code change clustering learning model} that learns to
untangle a~commit by clustering the code changes (represented by the
embeddings) into groups for different concerns.
%
While deterministic clustering criteria on the PDG, slices, or change
operations do not always produce the clusters that naturally map to
the boundaries between the concerns, a machine learning (ML) model is
expected to learn to cluster the changes, thus, untangling a
commit. Our ML model learns from the changes belonging to the same
concerns in the~version history. To build the training data for such
learning, we adapt Herzig {\em et al.}~\cite{kim-emse16}'s method to
mine the changes for the same concerns in the version history
(Section~\ref{overview:sec}).
%
To facilitate learning to cluster code changes, we develop a {\bf
  context-aware, graph-based, code change representation learning (RL)
  model}, leveraging
Label, Graph-based Convolution Network~\cite{label-gcn} to produce the
{\em contextualized embeddings for code changes}.

%\vspace{2pt}
Our clustering learning model and context-aware, graph-based RL model
have the following unique characteristics that facilitates the
untangling of code changes. First, we use Label-GCN to model the
changes and surrounding code by integrating both the versions before
and after changes in a multi-version program dependence graph,
$\delta$-PDG~\cite{flexeme-fse20}. $\delta$-PDG encodes the {\em
  program dependencies
  %\underline{program dependencies}
  among the changed and unchanged
  statements}. Second, to decompose the changes, we {\em explicitly
  represent the surrounding \underline{code context} of each
  change}. The explicit representation of the context could help
{\tool} learn the important features of a change (e.g., code
structures, data/control dependencies) to distinguish its concern
among others.
%
Third, {\tool} also considers an implicit relationship:
%an important feature for each change, that is,
the cloned code that is similar to the changed code under study.
%
The idea is that the {\em two \underline{cloned code} with similar
  logic might be changed in the same manner in the same concern in a
  commit}. Fourth, to untangle a commit, in our code change clustering
learning model, agglomerative clustering runs on {\em
  \underline{contextualized embeddings for code} \underline{changes}} that integrates richer, encoded information
than the PDG or program slices, thus, helping better distinguish the
concerns of the changes. Finally, we {\em \underline{adapt the
    agglomerative clustering algorithm}} into a {\em
  supervised-learning clustering model} with trainable parameters and
a loss function to adjust the parameters by comparing the predicted
clusters and the correct ones during training.

%We then apply agglomerative clustering on the vectors to produce the
%final clusters to untangle the commit. We expect that agglomerative
%clustering will be more effective as running on the embeddings with
%richer, encoded information than on the PDG or slices in
%distinguishing the concerns of the changes.

  %\textcolor{red}{We conduct several experiments to evaluate {\tool}...}

We have conducted several experiments to evaluate
    {\tool}. Our experimental results on a real-world C\# dataset with
    1,612 tangled commits show that {\tool} achieves the
    accuracy of 36.4\%, 28.6\%, 55.2\%, and 462.5\% relatively higher
    than the baseline approaches Flexeme~\cite{flexeme-fse20},
    $\delta$-PDG+CV~\cite{flexeme-fse20}, Herzig {\em et
      al.}~\cite{kim-emse16}, and Barnett {\em et
      al.}~\cite{barnett-icse15}. {\tool} can correctly cluster 39\%
    of the changed statements into correct concerns.
%
We also evaluated {\tool} in a Java dataset with 14k+ tangled
commits. The results show that {\tool} achieves 13.3\%--100\% accuracy
relatively higher than the state-of-the-art approaches
including SmartCommit~\cite{smartcommit-fse21}. Our sensitivity analysis
shows that all designed components in {\tool} contributes positively
to its accuracy. We show that the changed statements in the same
concerns are projected nearer to each other~than the ones in different
concerns, helping {\tool} better in change
clustering.  The key contributions of this work include:

{\bf 1. {\tool}: an ML-based, commit-untangling approach with a novel
  code change clustering learning model.} It is the first ML model
that learns to untangle the commit by learning to cluster the code
changes.  {\tool} learns from the changes belonging to the same
concern in the version history. We adapt agglomerative clustering
into a supervised-learning clustering model.

{\bf 2. A novel context-aware, graph-based representation learning for
  code changes.} We design GCN-based model to produce the {\em
  contextualized embeddings for the code changes}, that
integrates program dependencies, changes, contexts, and cloned code.
%Cloned code is also considered in untangling commits.

{\bf 3. Extensive empirical evaluation.} We evaluated {\tool} against
the state-of-the-art approaches for untangling commits to show its better
performance. Our tool and data are available at~\cite{utango-website}.

%\textcolor{red}{We have conducted several experiments to evaluate
%    {\tool}. Our experimental results on a real-world C\# dataset with
%    21k commits and 1,612 concerns show that {\tool} achieves the
%    accuracy of XX.X\%--XX\% and YY.Y\%--YY.Y\% relatively higher than
%    the state-of-the-art approaches Flexeme~\cite{flexeme-fse20},
%    Barnett {\em et al.}~\cite{barnett-icse15}, and Herzig {\em et
%      al.}~\cite{kim-emse16}, in cross-project and within-project
%    settings, respectively. {\tool} can correctly untangle the commits
%    by correctly clustering 39\% the changed statements into the
%    correct clusters/concerns.}
