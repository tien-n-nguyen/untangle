\section{Introduction}
\label{intro:sec}

During software evolution, developers make several changes over time
and commit them into a source code repository. The changes to the
source files that are committed to the repository at the same
transaction are often referred to as a {\em change set} or a {\em
  commit}. In an ideal world, each commit should be about one purpose
or concern regarding the programming task at hand.  Unfortunately, Tao
{\em et al.}~\cite{tao-fse12}, Kim {\em et
  al.}~\cite{kim-emse16,kim-msr13}, and Hill {\em et
  al.}~\cite{hill-tse12} have reported that many change sets or
commits tangle different concerns including the changes for
bug-fixing, refactoring, enhancements/improvements, or
documentation. Such change sets are called {\em tangled code changes}
or {\em tangled commits}~\cite{kim-emse16,kim-msr13}. The prior work
reported two reasons for tangled commits: time pressure in committing
the changes, and unclear boundaries between the concerns for code
changes~\cite{flexeme-fse20}.

Tangled commits pose several issues in software development. First,
they affect software quality in both hampering program
comprehension~\cite{tao-fse12} and reducing separation of concerns in
code changes~\cite{flexeme-fse20}. Second, the tangled commits
might contain the bug-fixing changes for one bug that are mixed with
the fixes for other bugs as well as different types of changes for
refactoring, enhancements, or
documentation~\cite{kim-emse16,kim-msr13,nguyen-issre13}. Those
tangled commits have negative impacts on the accuracy of bug
prediction or bug localization models that rely on the data mined from
the version histories~\cite{kim-emse16,kim-msr13}. Those models 
consider an entire commit as for fixing or non-fixing, thus,
are significantly affected by the tangled commits.

Recognizing the need of the tools that untangle, i.e., decompose a
commit into untangle changes, several researchers have proposed
different approaches that can be broadly classified into two
categories: {\em mining software repositories}, and {\em program
  analysis}.

First, earlier approaches leverage the {\em mining software
  repositories (MSR)} techniques. Herzig {\em et
  al.}~\cite{kim-msr13,kim-emse16} utilize a confidence voter
technique together with agglomerative clustering on the change
operations to untangle the commits.
%Each confidence voter is responsible for an important aspect
%including call-graphs, change couplings, data dependencies, and
%distance measures.
However, the voters are independent, thus, do not reflect well the
interdependency nature of program elements under change. In contrast,
Kirinuki {\em et al.}~\cite{higo-apsec16, higo-icpc14} rely on the
histories of the co-changes to split the tangled code changes before
they are committed. However, they do not consider the dependencies
among the changes such as data or control dependencies. Dias {\em et
  al.}~\cite{dias-saner15} also use confidence voters, but on the
fine-grained change events in an IDE. The scores are converted into
the similarity ones via a Random Forest Regressor, which are used in
the agglomerative clustering to partition the tangled changes.  The
second category of untangling approaches leverage the {\em static
  analysis} techniques. Roover {\em et al.}~\cite{roover-scam18} use
program slicing to segment a commit across a Program Dependency Graph
(PDG).  However, they are limited handling interprocedural and
cross-file dependencies. Barnett {\em et al.}~\cite{barnett-icse15}
utilize def-use chains, and cluster them. If the def-use chain all
fall into a method, it is considered as trivial, otherwise,
non-trivial. Because igoring the trivial clusters, it can miss tangled
concerns. To improve over that, Flexeme~\cite{flexeme-fse20} uses
multi-version PDG augmented with name/lexeme flows in the edges, and
applies Agglomerative Clustering using graph similarity on that graph
to untangle its commits.

Despite their successes, the state-of-the-art untangling commit
techniques still have limitations. First, the boundaries across the
concerns in a change set of a commit do not neccessarily and natually
map to a clustering criteria on the PDG (with/without name flows).
The concerns might be linked with multiple edges and a statement might
belong to multiple concerns. These points make the clustering
algorithms difficult to specify clustering criteria. Second, the goal
is to decompose the changes in a commit. However, the existing
approaches do not make a clear distinction of the context
of ...
